\documentclass[10pt,a4paper]{article}

\usepackage[utf8]{inputenc}
\usepackage[T1]{fontenc}
\usepackage[english]{babel}
\usepackage{amsmath,amsfonts,amssymb,mathrsfs,amsthm}

\usepackage[hidelinks]{hyperref}

\usepackage{fullpage}

\newtheorem{theorem}{Theorem}[section]
\newtheorem{corollary}[theorem]{Corollary}
\newtheorem{definition}[theorem]{Definition}
\newtheorem{example}[theorem]{Example}
\newtheorem{lemma}[theorem]{Lemma}
\newtheorem{proposition}[theorem]{Proposition}
\newtheorem{remark}[theorem]{Remark}

\DeclareMathOperator{\supp}{supp}

\begin{document}
  \section{Introduction}
  We consider the following cubic nonlinear Schr\"{o}dinger equation (NLS)
  \begin{equation}\label{nls-ep-nonl} % nls with epsilon nonlinearity
    \left\{
    \begin{aligned}
      & i\partial_t u = -\Delta u + \varepsilon^2 |u|^2 u, \quad (t,x) \in
      \mathbb R\times\mathbb R^d, \\
      & u|_{t=0} = \phi, \quad x \in \mathbb R^d,
    \end{aligned}
    \right.
  \end{equation}
  in dimension \( d \leq 3 \), where \(\phi\) is a given function, and \(
  \varepsilon \in (0,1] \) is a dimensionless parameter.%to characterize the size of
  %the initial data. 
  Note that, rescaling the amplitude by introducing \(w = \varepsilon u\),
  the original equation \eqref{nls-ep-nonl} with \(O(\varepsilon^2)\)-nonlinearity and 
  \(O(1)\)-initial data can be reformulated as the following equivelant NLS
  with \(O(1)\)-nonlinearity and \(O(\varepsilon)\)-initial data,
  \begin{equation}%\label{nls-ep-ini} %nls with epsilon initial value
    \left\{
    \begin{aligned}
      & i\partial_t w = -\Delta w + |w|^2 w, \quad (t,x) \in
      \mathbb R\times\mathbb R^d, \\
      & w|_{t=0} = \varepsilon \phi, \quad x \in \mathbb R^d.
    \end{aligned}
    \right.
  \end{equation}
  In this paper, we analysis the long time behavier of the numerical scheme
  given in~\cite{ORS21}.

  \section{Main theorem}
  As presented in~\cite{ORS21}, we choose the following scheme
  \begin{equation}\label{num-sol} %numerical solution
    \left\{
    \begin{aligned}
      & u^{n+1} = \Phi_K^{\tau}(u^n) := e^{i\tau\Delta} \left( u^n - i\tau\Pi_K
      \left( {(\Pi_K u^n)}^2 \varphi_1(-2i\tau\Delta)\Pi_K \bar{u}^n \right) \right), \\
      & u^0 = \Pi_K \phi,
    \end{aligned}
    \right.
  \end{equation}
  to approximate the solution \(u(t)\) at time \( t = t_{n+1} = t_n + \tau \),
  with \( \varphi_1(z) = \frac{e^z-1}{z} \) and the projection operator defined
  by the Fourier multiplier
  \begin{equation}
    \Pi_K = \chi^2 \left( \frac{-i\nabla}{K} \right),
  \end{equation}
  Here \(\chi\) is a smooth radial nonnegative bump function which is one on \(
  B(0,1) \) and supported in \( B(0,2) \), and \( K \geq 1 \) is considered as a
  parameter that will depend on \(\tau\). More precisely, we will choose \( K =
  \varepsilon^{-\beta}\tau^{-\frac\gamma2} \) with some \(\beta \geq 0\) and 
  \(\gamma \geq 1\).  
  %% TODO 解释选择gamma geq 1的原因

  Denoting \(u^K\) as the exact solution of the filtered PDE,
  \begin{equation}\label{nls-fil} % filtered NLS
    \left\{
    \begin{aligned}
      & i\partial_t u^K = -\Delta u^K + \varepsilon^2 \Pi_K|\Pi_K u^K|^2 \Pi_K u^K \\
      & u^K|_{t=0} = \Pi_K\phi,
    \end{aligned}
    \right.
  \end{equation}
  then we can bounded the total error by the following two parts
  \begin{equation}%\label{seperr}
    \| u(t_n) - u^n \|_{L^2} \leq \| u(t_n) - u^K(t_n) \|_{L^2} + 
    \| u^K(t_n) - u^n \|_{L^2}
  \end{equation}
  
  \begin{theorem}
    For every \( T > 0 \), \( \phi \in H^1 \) and \(\alpha > 0\) (\(\alpha \leq 4\) 
    if \(d=1\)), donote by \(u\) the exact solution of equation~\eqref{nls-ep-nonl} 
    and by \(u^n\) the sequence defined by the scheme~\eqref{num-sol}. Then, 
    there exist \(\tau_0 > 0\) such that for every step size \(\tau \in (0,\tau_0]\),
    every \( \varepsilon \in (0,1] \), we have the following error estimates:
    \begin{itemize}
      \item if \(d=1\), with the choise \(K=1/\varepsilon^\frac{4-\alpha}3\tau^\frac56\),
        \[ \|u(t_n)-u^n\|_{L^2} \lesssim \varepsilon^\frac{4-\alpha}3\tau^\frac56, \quad 
        0 \leq n\tau\varepsilon^\alpha \leq T \]
      \item if \(d=2\), with the choise \(K=1/\varepsilon^\beta\tau^\frac34\),
        for every \(\beta<1\) 
        \[ \|u(t_n)-u^n\|_{L^2} \lesssim_\beta \varepsilon^\beta \tau^\frac34, \quad 
        0 \leq n\tau\varepsilon^\alpha \leq T \]
      \item if \(d=3\), with the choise \(K=1/\varepsilon^\frac23\tau^\frac23\),
        \[ \|u(t_n)-u^n\|_{L^2} \lesssim \varepsilon^\frac23 \tau^\frac23|\log\tau+\log\varepsilon|^\frac23, 
        \quad 0 \leq n\tau\varepsilon^\alpha \leq T \]
    \end{itemize}
  \end{theorem}

  And we also have the following result, which is more practical. Particularly, for \(d=3\)
  with the help of scattering theory, we could have a global in time bound.
  \begin{theorem}
    For every \( T > 0 \), \( \phi \in H^1 \) and \(\alpha > 0\) (\(\alpha \leq 4\) 
    if \(d=1\)), donote by \(u\) the exact solution of equation~\eqref{nls-ep-nonl} 
    and by \(u^n\) the sequence defined by the scheme~\eqref{num-sol}. Then, 
    there exist \(\tau_0 > 0\) and \(\varepsilon_0>0 \) such that for every step size \(\tau \in (0,\tau_0]\),
    every \( \varepsilon \in (0,\varepsilon_0] \), we have the following error estimates:
    \begin{itemize}
      \item if \(d=1\), with the choise \(K=1/\tau^\frac56\),
        \[ \|u(t_n)-u^n\|_{L^2} \leq 2\|\phi\|_{H^1}\tau^\frac56 + C_{u_0,\alpha,T} \varepsilon^{2-\frac\alpha2}\tau^\frac56, \quad 
        0 \leq n\tau\varepsilon^\alpha \leq T \]
      \item if \(d=2\), with the choise \(K=1/\tau^\frac34\),
        for every \(\eta>0\) 
        \[ \|u(t_n)-u^n\|_{L^2} \leq 2\|\phi\|_{H^1}\tau^\frac56 + C_{u_0,\alpha,T,\eta} \varepsilon^{2-\eta} \tau^\frac34, \quad 
        0 \leq n\tau\varepsilon^\alpha \leq T \]
      \item if \(d=3\), with the choise \(K=1/\tau^\frac23\),
        \[ \|u(t_n)-u^n\|_{L^2} \leq 2\|\phi\|_{H^1}\tau^\frac23 + C_{u_0,\alpha} \varepsilon^{2} \tau^\frac23|\log\tau+\log\varepsilon|^\frac23, 
        \quad \forall n \geq 0 \]
    \end{itemize}
  \end{theorem}

  And by scattering theory we could also strengthen the result in \(d=3\) in~\cite{ORS21}. 
  \begin{theorem}
    For \(\varepsilon=1\) and the same assumptions as previous theorem in
    \(d=3\), we have
    \[ \|u(t_n)-u^n\|_{L^2} \lesssim \tau^\frac23|\log\tau|^\frac23, 
    \quad \forall n \geq 0, \]
    with the choice of \(K=\tau^{-\frac23}
  \end{theorem}
  Indeed this is different from the previous theorem, because in the previous
  theorem we only prove the global bound for \(\varepsilon< \varepsilon_0\),
  which may not take \(1\).

  \section{Continuous and discrete Strichartz estimates}

  First of all, let's recall the well known Strichartz estimates. We say that 
  \( (p,q) \) is admissible if \(p \geq 2\), \(q \geq 2\), \((p,q,d) \neq (2,\infty,2)\)
  and \( \frac2p + \frac{d}q = \frac{d}2 \). We denote \(p'\)
  and \(q'\) be the dual of \(p\) and \(q\), i.e., \(\frac1p+\frac1{p'} = 1\) 
  and \(\frac1q+\frac1{q'} = 1\). The following three results have been proved
  in~\cite{ORS21}.

  \begin{theorem}
    For every \((p,q)\) admissible, there exists \(C > 0\) such that for every 
    \( f \in L^2 \) and \( F \in L^{p'} L^{q'} \)
    \begin{eqnarray}
      \label{Tc} & & \|e^{it \Delta} f \|_{L^p{L^q}} \leq C \| f\|_{L^2} \\
      \label{T*c} & & \left \| \int_{\mathbb R} e^{-is \Delta } F(s,\cdot)\, ds\right\|_{L^2} 
      \leq C \|F \|_{L^{p'} L^{q'}}.
    \end{eqnarray}
    Moreover, for every \((p_1, q_1)\) and \((p_2, q_2)\) admissible, there exists
    \(C > 0\) such that for every \(F \in L^{p_2'} L^{q'_2}\), we have
    \begin{equation}\label{TT*c} 
      \left\| \int_{-\infty}^t e^{i ( t-s ) \Delta } F(s,\cdot) \, ds \right\|_{L^{p_{1}} L^{q_{1}}}
      \leq  C \|F \|_{L^{p_{2}'} L^{q'_{2}}}.
    \end{equation}
  \end{theorem}
  %% TODO 给出上述定理的证明来源。

  We also have the discrete versions of these estimates for the group 
  \begin{equation}\label{SK}
    S_K(t) = e^{it\Delta} \Pi_K = \Pi e^{it\Delta}.
  \end{equation}

  \begin{theorem}\label{thmDSEold} % discrete Strichartz estimates
    For every \((p,q)\) admissible with \(p>2\), there exists \(C>0\) such that
    for every \(K\) and \(\tau\) satisfying \(K\tau^\frac12 \geq 1\), all \(
    f \in L^2 \) and all \(F \in l^{p'}_\tau L^{q'}\), we have 
    \begin{eqnarray}
      \label{To} & & \left\| S_{K}(n \tau) f\right\|_{l^p_\tau L^q} \leq C{(K\tau^\frac12)}^{\frac2p} \|f\|_{L^2} \\
      \label{T*o} & & \left\| \tau\sum_{n \in \mathbb{Z}}  S_K (-n\tau) F_n \right\|_{L^2} \leq C {(K\tau^\frac12)}^\frac2p \|F\|_{l^{p'}_\tau L^{q'}}.
    \end{eqnarray}
    For every \((p_1, q_1)\) and \((p_2, q_2)\) admissible with \(p_1>2\),
    \(p_2>2\), there exists \(C > 0\) such that for every \(K\) and \(\tau\) 
    satisfying \(K\tau^\frac12 \geq 1\), all \( s \in [-3,3] \) and all 
    \(F \in L^{p_2'} L^{q'_2}\), we have
    \begin{equation}\label{TT*o}
      \left\| \tau \sum_{k = -\infty}^{n-1}  S_K ((n-k+s)\tau) F_k \right\|_{l^{p_1}_\tau L^{q_1}}
      \leq C (K\tau^\frac12)^{ \frac2{p_1} + \frac2{p_2} }  \|F\|_{l^{p_2'}_\tau L^{q_2'}}.
    \end{equation}
  \end{theorem}

  %% TODO 指出 这里我们去掉了端点,给出证明来源

  \begin{corollary}\label{corDSEold}
    For every $(p,q)$ admissible with $p>2$, there exists $C>0$ such that for every $0<\tau \leq 1$ and $K = \tau^{-\frac\gamma2}$, $\gamma \geq 1$,  we have
    \begin{equation}
    \label{Tlossold}
    \left\| S_{K}(n \tau) f\right\|_{l^p_{\tau}L^q} \leq C    \|f \|_{H^{ \frac2p(1 - \frac1\gamma) }}\qquad \text{for all \ $f\in H^{ {2 \over p}( 1 - {1 \over \gamma} ) }$}. 
    \end{equation}
    For every $(p, q)$ admissible with  $p>2$, there exists $C>0$ such that for every  $0<\tau \leq 1$, $K = \tau^{-{\gamma \over 2}}$,  $\gamma \ge 1$  and $s \in [-8,8]$ we have
    \begin{equation}
    \label{TT*lossold}
    \left\| \tau  \sum_{k=-\infty}^{n-1}   S_{K} ((n-k+ s) \tau)  F_{k} \right\|_{l^{p}_{\tau}L^{q} } \leq C   \| F\|_{l^1_\tau  H^{ {2 \over p}( 1 - {1 \over \gamma} ) } }\qquad 
    \text{for all \ $F\in l^1_\tau  H^{ {2 \over p}( 1 - {1 \over \gamma} ) }$}. 
    \end{equation}
  \end{corollary}
  Note that, since $\tau^{1 \over p}\| \Pi_{K} f\|_{L^q} \leq  \left\| S_{K}(n \tau) f\right\|_{l^p_{\tau}L^q}$ 
  the estimate \eqref{Tloss} also encodes the modified Sobolev estimate
  \begin{equation}
    \label{sobmodold}
    \tau^{1 \over p} \| \Pi_{K} f\|_{L^q} \leq C \|f \|_{H^{ {2 \over p}( 1 - {1 \over \gamma} ) }}.
  \end{equation}

  Indeed, we can generalize the above two theorem, and have the following, for
  \(K = \varepsilon^{-\beta\gamma} \tau^{-\frac\gamma2}\). And for all exponent
  pairs \((p,q)\) with \( \frac2p + \frac{d}q \leq \frac{d}2\) if \(p\neq2\) and
  \( \frac2p + \frac{d}q < \frac{d}2\) if \(p=2\). We call such a pair weakly
  admissible.

  \begin{theorem}\label{thmDSE} % discrete Strichartz estimates
    For every exponent pair \((p,q)\) weakly admissible, there exists \(C>0\) such that
    for every \(K\) and \(\tau\) satisfying \(K\tau^\frac12 \geq 1\), all \(
    f \in L^2 \) and all \(F \in l^{p'}_\tau L^{q'}\), we have 
    \begin{eqnarray}
      \label{T} & & \left\| S_{K}(n \tau) f\right\|_{l^p_\tau L^q} \leq C{(K\tau^\frac12)}^{d(\frac12-\frac1q)} \|f\|_{L^2} \\
      \label{T*} & & \left\| \tau\sum_{n \in \mathbb{Z}}  S_K (-n\tau) F_n \right\|_{L^2} 
      \leq C {(K\tau^\frac12)}^{d(\frac12-\frac1q)} \|F\|_{l^{p'}_\tau L^{q'}}.
    \end{eqnarray}
    For every \((p_1, q_1)\) and \((p_2, q_2)\) weakly admissible,
    there exists \(C > 0\) such that for every \(K\) and \(\tau\) 
    satisfying \(K\tau^\frac12 \geq 1\), all \( s \in [-3,3] \) and all 
    \(F \in L^{p_2'} L^{q'_2}\), we have
    \begin{equation}\label{TT*}
      \left\| \tau \sum_{k = -\infty}^{n-1}  S_K ((n-k+s)\tau) F_k \right\|_{l^{p_1}_\tau L^{q_1}}
      \leq C (K\tau^\frac12)^{d(1-\frac1{q_1}-\frac1{q_2})}  \|F\|_{l^{p_2'}_\tau L^{q_2'}}.
    \end{equation}
  \end{theorem}

  \begin{corollary}\label{corDSE}
    For every \((p,q)\) weakly admissible, there exists \(C>0\) such that 
    for every \(0<\tau \leq 1\), \( 0 < \varepsilon \leq 1 \) and 
    \(K = \varepsilon^{\beta\gamma}\tau^{-\frac\gamma2}\), \(\gamma \geq 1\), 
    \(\beta \geq 0\),  we have
    \begin{equation}
    \label{Tloss}
      \left\| S_{K}(n \tau) f\right\|_{l^p_\tau L^q} \leq \varepsilon^{-\beta d(\frac12-\frac1q)} 
      \|f \|_{H^{(1-\frac1\gamma)d(\frac12-\frac1q)}} \qquad \text{for all \(f \in H^{(1-\frac1\gamma)d(\frac12-\frac1q)}\)}. 
    \end{equation}
    For every \((p_1, q_1)\) and \((p_2, q_2)\) weakly admissible, there exists \(C>0\) such that for 
    for every \(0<\tau \leq 1\), \( 0 < \varepsilon \leq 1 \) and 
    \(K = -\varepsilon^{\beta\gamma}\tau^{-\frac\gamma2}\), \(\gamma \geq 1\), 
    \(\beta \geq 0\) and \(s \in [-8,8]\) we have,
    \begin{equation}
    \label{TT*loss}
      \left\| \tau  \sum_{k=-\infty}^{n-1} S_{K} ((n-k+s) \tau) F_k \right\|_{l^p_\tau L^q} 
      \leq \varepsilon^{-\beta d(\frac12-\frac1q)} \|F\|_{l^{p_2}_\tau B^{(1-\frac1\gamma)d(1-\frac1{q_1}-\frac1{q_2})}_{q_2',2}}\qquad 
      \text{for all  \(F\in l^{p_2}_\tau B^{(1-\frac1\gamma)d(1-\frac1{q_1}-\frac1{q_2})}_{q_2',2}\) },
    \end{equation}
    where we use \(B^s_{p,q}\) to denote inhomogeneous Besov space.
  \end{corollary}
  Note that, since \(\tau^\frac1p\| \Pi_{K} f\|_{L^q} \leq  \left\| S_{K}(n \tau) f\right\|_{l^p_{\tau}L^q}\) 
  the estimate~\eqref{Tloss} also encodes the modified Sobolev estimate
  \begin{equation}
    \label{sobmod}
    \tau^\frac1p \|\Pi_K f\|_{L^q} \leq \varepsilon^{-\beta d(\frac12-\frac1q)} \|f\|_{H^{(1-\frac1\gamma)d(\frac12-\frac1q)}}.
  \end{equation}



  %% TODO 给出Strichartz估计的表述和admissible的表述
  % 给出离散的估计
  
  \begin{proposition}
    Let \(\phi \in H^1\), \( d \leq 3 \) and \( (p,q) \) admissible with
    \( p > 2 \). Let \( u \) be the solution of~\eqref{nls-ep-nonl} and \(u^K\)
    be the solution of~\eqref{nls-fil}. We have for every \( \varepsilon \in 
    (0,1] \),
    \begin{equation}
      \|u\|_{L^\infty H^1} \lesssim 1 \quad
      \|u^K\|_{L^\infty H^1} \lesssim 1.
    \end{equation}
  \end{proposition}
  
  \begin{proof}
    By using \( \varepsilon \leq 1 \), the estimate 
    \[ \|u\|_{L^\infty H^1} \lesssim 1 \]
    is the consequence of the conservation laws. And for \(u^K\), we have the
    similar conservation laws for 
    \[ H[u(t)] = \int_{\mathbb R^d} \frac12|\nabla u^K(t,x)|^2 + 
    \frac{\varepsilon^2}2 |\Pi_K u(t,x)|^4 \,\mathrm{d}x, \quad M[u(t)] = 
    \int_{\mathbb R^d} |u^K(t,x)|^2 \,\mathrm{d}x. \]
    The proof follows exactly the proof of the classical conservation laws, then
    we have the save estimates uniformly in \(\varepsilon\)
    \[ \|u^K\|_{L^\infty H^1} \lesssim 1.\]
  \end{proof}


  Now we consider a subset of all admissible \((p,q)\),
  \begin{equation}\label{lambdad}
    p>2, (p,q)\text{ admissible, } p<4 \text{ if } d=3 \text{ and }
    (p,q)= (4,\infty) \text{ if } d=1\}
  \end{equation}
  We ask \(p<4\) when \(d=3\) because in such situation we have \(q>3\), hence
  \(W^{1,q} \subset L^\infty\).
  In dimension 1, we have by H\"older's inequality
  \[ \|\nabla u|u|^2\|_{L^\frac43_I L^1} + \|u|u|^2\|_{L^\frac43_I L^1} \leq 
  |I|^\frac12 \|u\|_{L^4_I W^{1,\infty}} \|u\|^2_{L^\infty_I L^2}. \]
  In dimension 2 and 3, similarly, for \((\bar{p},\bar{q})\) admissible we can write
  \begin{equation}\label{XT}
    \|\nabla u |u|^2\|_{L^{\bar{p}'}_I L^{\bar{q}'}} \leq \left\| 
    \|\nabla u\|_{L^q} \|u\|_{L^r}^{2\gamma} \|u\|_{L^q}^{2(1-\gamma)}
    \right\|_{L^{\bar{p}'}_I} \leq |I|^\theta \|\nabla u\|_{L^p_I L^q} 
    \|u\|_{L^\infty_I L^r}^{2\gamma} \|u\|_{L^p_I L^q}^{2(1-\gamma)},
  \end{equation}
  and the same estimates by substituting \(\nabla u\) with \(u\),
  if \( \frac1q + \frac{2\gamma}r + \frac{2(1-\gamma)}q = \frac1{\bar{q}'} \)
  and \( \frac1p + \frac{2(1-\gamma)}p + \theta = \frac1{\bar{p}'} \). We may
  choose a suitable \(r\) such that \(L^r \subset H^1\). Indeed, if \(d=2\),
  we can choose \((\bar{p},\bar{q})=(\infty,2)\), \(p=\frac2{2-\theta}\),
  \(q=\frac2\theta\), \(r=\frac2{1-2\theta}\) and \(\gamma=\frac12\), and if
  \(d=3\), we can choose \((\bar{p},\bar{q})=(\frac1{\theta(1-\theta)},\frac6{3-4\theta(1-\theta)})\), 
  \(p=\frac2{2-\theta}\), \(q=\frac6{2\theta+1}\), \(r=6\) and \(\gamma=\frac{1+2\theta}2\).

  \begin{proposition}
    Let \(\phi \in H^1\), \( d \leq 3 \) and \((p,q)\) satisfying~\eqref{lambdad}. 
    Let \( u \) be the solution of~\eqref{nls-ep-nonl}, we
    have the following estimates for every \( \varepsilon \in (0,1] \) and every
    time interval \( I = [t_1,t_2] \)
    \[ \|u\|_{L^p_I W^{1,q}} \lesssim
    \langle\varepsilon^\alpha |I|\rangle^{\frac1p} \]
    where \( \frac1\alpha = \frac12 - \frac1p \). Moreover, let \( u^K \) be the
    solution of~\eqref{nls-fil}, \(u^K\) also satisfies the above estimates. 
  \end{proposition}

  \begin{remark}
    %% TODO
    % 加一些话:
    % 为什么要用 <>来估计
    % 虽然在某些p,q选取中这里的alpha不是最优的,但是对于我们的目的已经足够了
    %
    By H\"older's inequality, we have
    \begin{equation}\label{uL2Linfty}
      \varepsilon^2 \|u\|_{L^2_I L^\infty}^2 \leq \varepsilon^2 |I|^{2(\frac12-\frac1p)}\|u\|_{L^p_I W^{1,q}}^2 
      \lesssim_{\alpha} {(\varepsilon^\alpha |I|)}^\frac2\alpha {\langle\varepsilon^\alpha |I|\rangle}^{1-\frac2\alpha},
    \end{equation}
    where \(\frac1\alpha=\frac12-\frac1p\). If \(d=1\), then \(\alpha=4\),
    if \(d=2,3\), we can choose \(p\) close to \(2\) and we get any large enough
    \(\alpha>0\).
  \end{remark}
  
  \begin{proof}
    Let's recall the well-known argument that proves the local well-posedness of
    NLS. We consider \(\Phi_I: X_I \to X_I \) with \(X_I=L_I^\infty H^1
    \cap L^p_I W^{1,q}\), \(\|\cdot\|_{X_I} = \|\cdot\|_{L^\infty_I H^1} +
    \|\cdot\|_{L^p_I W^{1,q}}\) and 
    \[ \Phi_I(u) = e^{it\Delta} u(t_1) - i\varepsilon^2 \int_{t_1}^{t_2} e^{i(t-s)\Delta}
    f(s)\,\mathrm{d}s, \quad f(s)=|u(s)|^2u(s). \]
    Ihen by Strichartz estimates~\eqref{Tc}, \eqref{TT*c} and the discuss~\eqref{XT} above ,
    \begin{align}
      \|\Phi_I(u)\|_{X_I} & \lesssim \|u(t_1)\|_{H^1} + \varepsilon^2
      \|f\|_{L^{\bar{p}'}_I W^{1,\bar{q}'}} \\
      & \lesssim \|u\|_{L^\infty H^1} + \varepsilon^2 |I|^\theta \|u\|_{X_I}^3
    \end{align}
    and 
    \begin{equation}
      \|\Phi_I(u)-\Phi_I(v)\|_{X_I} \lesssim \varepsilon^2 |I|^\theta 
      \|u-v\|_{X_I} (\|u\|_{X_I}^2 + \|v\|_{X_I}^2),
    \end{equation}
    where \( \theta = 1 - \frac2p \). By the fix point argument, we could have
    the estimate uniformly in \(\varepsilon\),
    \begin{equation}\label{smallI}
      \|u\|_{X_I} \lesssim 1,\quad \text{for }\varepsilon^\frac2\theta |I| < \delta
    \end{equation}
    for some \(\delta\) depending on \(\|u_0\|_{H^1}\). And for large \(|I|\),
    we can iterate the above argument on \([t_1,t_1+\delta/(\varepsilon^{2/\theta})]\), 
    \([t_1+\delta/(\varepsilon^{2/\theta}),t_1+2\delta/(\varepsilon^{2/\theta})]\), \(
    \ldots\), for some \(\delta\) depending on \(\|u\|_{L^\infty H^1}
    \lesssim_{\|\phi\|_{H^1}} 1 ,\)
    \begin{equation}\label{largeI}
      \|u\|_{L^\infty_I W^{1,q}} \lesssim (\varepsilon^\frac2\theta |I|)^\frac1p,
    \quad \text{for } \varepsilon^\frac2\theta |I| < \delta.
    \end{equation}
    Combining~\eqref{smallI} and~\eqref{largeI}, we finally get 
    \[ \|u\|_{L^p_I W^{1,q}} \lesssim
    \langle\varepsilon^\alpha |I|\rangle^{\frac1p},\]
    where \(\frac1\alpha=\frac\theta2=\frac12 - \frac1p.\) And the estimate for
    \(u^K\) follows exactly the same lines of the proof above.
  \end{proof}
  %\begin{definition}
  %  We say that triple \((p,q,\alpha)\) verifies property {\bf (H)} if \((p,q)\)
  %  is admissible, \(\frac1\alpha = \frac12 - \frac1p\) and \(p=4\) if \(d=1\),
  %  \(p<4\) if \(d=3\).
  %\end{definition}
  %\begin{remark}
  %  We ask \(p<4\) when \(d=3\) because in such situation we have \(q>3\), hence
  %  \(W^{1,q} \subset L^\infty\).
  %  In dimension 1, \((p,q,\alpha)=(4,\infty,4)\),
  %  we have by H\"older's inequality
  %  \[ \|\nabla u|u|^2\|_{L^\frac43_T L^1} + \|u|u|^2\|_{L^\frac43_T L^1} \leq 
  %  T^\frac12 \|u\|_{L^4_T W^{1,\infty}} \|u\|^2_{L^\infty_T L^2}. \]
  %  In dimension 2 and 3, similarly, we can write
  %  \begin{equation}
  %    \|\nabla u |u|^2\|_{L^{\bar{p}'}_T L^{\bar{q}'}} \leq \left\| 
  %    \|\nabla u\|_{L^q} \|u\|_{L^r}^{2\gamma} \|u\|_{L^q}^{2(1-\gamma)}
  %    \right\|_{L^{\bar{p}'}_T} \leq T^\theta \|\nabla u\|_{L^p_T L^q} 
  %    \|u\|_{L^\infty_T L^r}^{2\gamma} \|u\|_{L^p_T L^q}^{2(1-\gamma)},
  %  \end{equation}
  %  if \( \frac1q + \frac{2\gamma}r + \frac{2(1-\gamma)}q = \frac1{\bar{q}'} \)
  %  and \( \frac1p + \frac{2(1-\gamma)}p + \theta = \frac1{\bar{p}'} \). We may
  %  choose a suitable \(r\) such that \(L^r \subset H^1\). Indeed, if \(d=2\),
  %  we can choose \(\bar{p}=\infty\), \(\bar{q}=2\), \(p=\frac2{2-\theta}\),
  %  \(q=\frac2\theta\), \(r=\frac2{1-2\theta}\) and \(\gamma=\frac12\), and if
  %  \(d=3\), we can choose \(bar\)
  %\end{remark}

  \begin{corollary}\label{coru-uKLinftyL2}
    For \( d \leq 3 \), \( \phi \in H^1 \), for every \( \alpha > 0 \) (\( \alpha
    \leq 4 \) if \( d = 1 \)), we have the estimate for every \( \varepsilon \in
    (0,1] \),
    \begin{equation}
      \|u-u^K\|_{L^\infty_{T/\varepsilon^\alpha} L^2} \lesssim_\alpha
      \frac1K.
    \end{equation}
  \end{corollary}
  \begin{proof}
    By using Duhamel's formula, we have
    that for every interval \(I = [t_1 , t_2]\),
    \begin{multline}
      u(t_2) - u^K(t_2) = e^{it\Delta}(u(t_1) - u^K(t_1))  
      - i \varepsilon^2 \int_{t_1}^{t_2} e^{i(t-s)\Delta} \Pi_K (|u|^2u 
      - |\Pi_K u|^2\Pi_K u) \,\mathrm{d}s \\ 
      - i \varepsilon^2 \int_{t_1}^{t_2} e^{i(t-s)\Delta} \Pi_K (|\Pi_K u|^2\Pi_K u 
      - |\Pi_K u^K|^2\Pi_K u^K) \,\mathrm{d}s  
      - i \varepsilon^2 \int_{t_1}^{t_2} e^{i(t-s)\Delta} (1 - \Pi_K)(|u|^2u) 
      \,\mathrm{d}s .
    \end{multline}
    We then obtain that
    \begin{multline}\label{u-uKLinftyL2}
      \|u-u^K\|_{L^{\infty}_{I}L^2} - \|(u-u^K)(t_1)\|_{L^2} \lesssim
      \varepsilon^2 \| u - \Pi_K u \|_{L^{\infty}_{I}L^2} ( \|u\|^2_{L^2_{I}L^{\infty}} + 
      \|u^K\|^2_{L^2_{I}L^{\infty}} ) \\
      + \varepsilon^2 \| u - u^K \|_{L^{\infty}_{I}L^2} ( \|u\|^2_{L^2_{I}L^{\infty}} + 
      \|u^K\|^2_{L^2_{I}L^{\infty}} ) + \varepsilon^2 \| u - \Pi_K u \|_{L^{\infty}_{I}L^2} 
      \|u\|^2_{L^2_{I}L^{\infty}}.
    \end{multline}
    By conservation laws, we have the estimate
    \begin{equation}\label{u-PiKu}
      \|u-\Pi_K u\|_{L^\infty L^2} \lesssim \frac1K \|u\|_{L^\infty H^1}
      \lesssim \frac1K
    \end{equation}
    By plugging~\eqref{u-PiKu} and~\eqref{uL2Linfty} into~\eqref{u-uKLinftyL2},
    we obtain that
    \[ \|u-u^K\|_{L^{\infty}_{I}L^2} - \|(u-u^K)(t_1)\|_{L^2} 
    \lesssim_{\alpha} \frac1K 
    {(\varepsilon^\alpha |I|)}^\frac2\alpha {\langle\varepsilon^\alpha |I|\rangle}^{1-\frac2\alpha} + 
    {(\varepsilon^\alpha |I|)}^\frac2\alpha {\langle\varepsilon^\alpha |I|\rangle}^{1-\frac2\alpha}
    \|u-u^K\|_{L^{\infty}_{I}L^2}. \]
    Consequently, we can choose \(\delta\) small enough such that for every 
    \(\varepsilon^\alpha |I| < \delta\)
    \[ \|u-u^K\|_{L^{\infty}_{I}L^2} - \|(u-u^K)(t_1)\|_{L^2} 
    \lesssim_\alpha \frac1K. \]
    Finally, ue can iterate the above argument to get 
    \[ \|u-u^K\|_{L^\infty_{T/\varepsilon^\alpha} L^2} \lesssim_\alpha
    \frac1K.\]
  \end{proof}

  \section{Discrete Strichartz estimates for exact solution}

  %% TODO 加一些解释性的话
  % property H 和 相应的计算
  %

  Let \(K=\tau^{-\frac\gamma2}\) for some \(\gamma\geq1\), \(\gamma<2\) in
  dimension \(3\) and \((p,q)\) be admissible. We suppose that \((p,q)=(4,\infty)\) if
  \(d=1\) and \(2<p<2+\delta\) for some small \(\delta>0\) depending on \(\alpha\).
  Then there always exists a \(\sigma\) such that 
  \begin{equation}\label{prpH}
    \sigma q > d, \quad \sigma + \frac2p \left( 1-\frac1\gamma \right) \leq 1.
  \end{equation}
  Indeed, when \(d=1\), we have \(p=4,q=\infty\) then every \(\sigma \in
  (0,\frac12] \) and every \(\alpha\geq1\) satisfying~\eqref{prpH}. In dimension
  \(2\), the existence of \(\sigma\) satisfying~\eqref{prpH} is equivelant to
  \( \frac dq +\frac2p (1-\frac1\alpha) \leq 1 \), which is always true when
  \(\alpha \geq 1\) and \((p,q)\) is admissible. In dimension \(3\), the claim
  above is equivelant to \(p\alpha < 4\). Consequently, if \(\alpha<2\), we can
  find small \(\delta>0\) such that this is satisfied.


  The statement and the proof of the following theorem is wrong, I will fix it
  later!!!
  Fortunately, the estimate of the norm \(l^2L^\infty\) can still hold.

  \begin{proposition}\label{uKlpLq}
    Let \(K = \tau^{-\frac\gamma2}\) for some \(\gamma\geq1\), \(\gamma<2\)
    in dimension \(3\). Furthur let triple \((p,q,\sigma)\) verify~\eqref{prpH}.
    Then for every \( \tau \in (0,1] \) and every \(\hat{s} \in 
    [-2\tau,2\tau] \), ue have the estimate
    \begin{equation}
      \sup_{s\in[0,\tau]} \|e^{i\hat{s}\Delta}u^K(t_k+s)\|_{l^p_{\tau,N} W^{\sigma,q}}
      \lesssim \langle\varepsilon^\alpha N\tau\rangle^\frac1p,
    \end{equation}
    where \(\frac1\alpha = \frac12 - \frac1p\).
  \end{proposition}
  
  \begin{remark}\label{remuKlpLq}
    As we done in the continuous case, under the above assumptions and in the
    particular case when \(\hat{s}=0\), we get by Sobolev embedding and H\"older's
    inequality that
    \begin{equation}\label{uKl2Linfty}
      \varepsilon \sup_{s\in[0,\tau]} \|u^K(t_k+s)\|_{l^2_{\tau,N} L^\infty}
      \lesssim_{\alpha} {(\varepsilon^\alpha N\tau)}^\frac1\alpha {\langle\varepsilon^\alpha N\tau\rangle}^{\frac12-\frac1\alpha},
    \end{equation}
    for \(\alpha\leq4\) if \(d=1\) and every \(\alpha>0\) if \(d=2,3\).
    
    We can easily perform the same argument in the proof of the proposition to get
    the estimates
    \[ \sup_{s\in[0,\tau]} \|\left(e^{i\hat{s}\Delta}u^K(t_k+s)\right)_{n\leq k\leq n+N}\|
    _{l^p_{\tau} W^{\sigma,q}} \lesssim \langle\varepsilon^\alpha N\tau\rangle^\frac1p \]
    and 
    \[
      \varepsilon \sup_{s\in[0,\tau]} \|\left(u^K(t_k+s)\right)_{n\leq k\leq n+N}\|
      _{l^2_{\tau} L^\infty} \lesssim_{\alpha} {(\varepsilon^\alpha N\tau)}^\frac1\alpha
      {\langle\varepsilon^\alpha N\tau\rangle}^{\frac12-\frac1\alpha}.\]
  \end{remark}

  \begin{proof}
    We first prove the estimate for \(\hat{s}=0\).

    Let us first observe that by elliptic regularity, we have for every time
    \(t\),
    \[ \tau^{\frac{1}{p}} \|u^K(t)\|_{W^{\sigma,q}} \lesssim \tau^{\frac{1}{p}}
    \|(I-\Delta)^{\frac{\sigma}{2}} u^K(t)\|_{L^q}, \]
    therefore, since \(\Pi_{2K}u^K=u^K\), we can use the modified Sobolev
    estimate~\eqref{sobmod} to get 
    \begin{equation}\label{l0usq}% estimate for l_0 W^sigma,q
      \tau^{\frac{1}{p}} \|u^K(t)\|_{W^{\sigma, q}} \lesssim 
      \tau^{\frac{1}{p}} \|(I-\Delta)^{\frac{\sigma}{2}} u^K(t)\|_{L^q} 
      \lesssim \|u^K(t)\|_{H^{\sigma_1}} 
      \lesssim_{\|\phi\|_{H^1}} 1,
    \end{equation}
    where \(\sigma_1 = \sigma + \frac2p(1-\frac1\gamma) \leq 1\).
    We use Duhamel's formula to get that for every \( 0 \leq n \leq N \) and 
    \( s \in [0,\tau] \),
    \[ u^K(t_n+s) = e^{i(n+1)\tau\Delta} u^K(s-\tau) - i\varepsilon^2 
    \int_{s-\tau}^{s+t_n} e^{i(t_n+s-\tilde{s})\Delta}\Pi_K
    \left(|\Pi_K u^K|^2 \Pi_K u^K\right) (\tilde{s}) \mathrm{d} \tilde{s}. \]
    We rewrite as 
    \begin{equation}\label{disc-str-est-duham}
      u^K(t_n+s) = S_{2K}((n+1)\tau) u^K(s-\tau) - i\varepsilon^2 
      \sum_{k=0}^n \int_{s-\tau}^{s} S_K(t_{n-k}+s-\tilde{s}) \left(|\Pi_K u^K|^2 \Pi_K u^K\right) (t_k+\tilde{s}) \mathrm{d} \tilde{s}.
    \end{equation}
    We get by Strichartz estimate, % TODO link to eqref
    \begin{equation}
      \|u^K(t_k+s)\|_{l^p_{\tau,n}W^{\sigma,q}} \lesssim \|u^K(s-\tau)\|_{H^1} + \tau^{-1}
      \varepsilon^2 \int_{s-\tau}^s \left\|\left(|\Pi_K u^K|^2 \Pi_K u^K\right) 
      (t_k+\tilde{s})\right\|_{l^1_{\tau,n}H^{\sigma_1}} \mathrm{d} \tilde{s}.
    \end{equation}
    Next, we can use that 
    \[ \left\|\left(|\Pi_K u^K|^2 \Pi_K u^K\right) (t_k+\tilde{s})\right\|_{l^1_{\tau,n}H^{\sigma_1}}
    \leq \left\|\left(|\Pi_K u^K|^2 \Pi_K u^K\right) (t_k+\tilde{s})\right\|_{l^1_{\tau,n}H^1}
    \leq \|u^K\|_{L^{\infty}H^1}\|u^K(t_k+\tilde{s})\|_{l^2_{\tau,n}L^{\infty}}^2. \]
    Since by Sobolev embedding and H\"older inequality, we have 
    \[ \|u^K(t_k+\tilde{s})\|_{l^2_{\tau,n}L^{\infty}}^2 \lesssim
    (n\tau)^{1-\frac{2}{p}}\|u^K(t_k+\tilde{s})\|_{l^p_{\tau,n}W^{\sigma,q}}^2.\]
    For \( \tilde{s} < 0 \), we have by~\eqref{l0usq}
    \[ \|u^K(t_k+\tilde{s})\|_{l^p_{\tau,n}W^{\sigma,q}} \leq
    \|u^K(t_k+\tilde{s})\|_{l^p_{\tau,(n-1)}W^{\sigma,q}} + \tau^{\frac{1}{p}} \|u^K(\tilde{s}-\tau)\|_{W^{\sigma, q}}
    \leq \|u^K(t_k+\tilde{s})\|_{l^p_{\tau,n}W^{\sigma,q}} + C(\|\phi\|_{H^1}).
    \]
    Combining all estimates above, we finally get 
    \[ \|u^K(t_k+s)\|_{l^p_{\tau,n}W^{\sigma,q}} \lesssim
    C(\|\phi\|_{H^1}) + \varepsilon^2 \frac{s-\tau}{\tau} C(\|\phi\|_{H^1})
    (n\tau)^{1-\frac{2}{p}} + \varepsilon^2 (n\tau)^{1-\frac{2}{p}}
    \sup_{\tilde{s} \in [0,\tau]}\|u^K(t_k+\tilde{s})\|_{l^p_{\tau,n}W^{\sigma,q}} \]
    Consequently, we could find a generic \( \delta \) small enough such that
    for every \( \varepsilon^\alpha n \tau < \delta \),
    \[ \|u^K(t_k+\tilde{s})\|_{l^p_{\tau,n}W^{\sigma,q}} \lesssim_{\|\phi\|_{H^1}} 1. \]
    Finally, by interating, we get the estimate we want
    \[ \sup_{s\in[0,\tau]} \|u^K(t_k+s)\|
      _{l^p_{\tau,N}W^{\sigma,q}} \lesssim \langle\varepsilon^\alpha N\tau\rangle^\frac1p.\]
    To get the estimate in \( \hat{s} \neq 0 \), we can just apply
    \( e^{i\hat{s}\Delta} \) to~\eqref{disc-str-est-duham}. All subseqwent
    arguments are the same. 
  \end{proof}

  \begin{corollary}\label{corphiuKlpLq}
    Under the same assumption of the previous proposition, we also have 
    \begin{equation}\label{phiuKlpLq}
      \|\varphi_1(-2i\tau\Delta) u^K(t_k)\|_{l^p_{\tau,N} W^{\sigma,q}} \lesssim 
      \langle\varepsilon^\alpha N\tau\rangle^\frac1p,
    \end{equation}
    where \(\frac1\alpha = \frac12 - \frac1p\).
  \end{corollary}

  \begin{proof}
    We first note that we can decompose
    \[
      \varphi_1(-2i\tau\Delta) u^K(t_k) = \varphi_1(-2i\tau\Delta) (1 - \Pi_{\tau^{-{1 \over 2}}})u^K(t_k) + \varphi_1(-2i\tau\Delta)  \Pi_{\tau^{-{1 \over 2}}}u^K(t_k).
    \]
    By using Lemma \ref{lemfilter1}, we have that  the multiplier $ \varphi_{1}(-2i \tau \Delta)  \Pi_{\tau^{-{1 \over 2}}} $ is continuous on $L^q$ for every $q$ with norm uniform in $\tau$. 
    Therefore, we get from Proposition \ref{uKlpLq} that
    $$
    \| \varphi_{1}(-2i \tau \Delta) { \Pi_{\tau^{-{1 \over 2}}}} u^K(t_{k})\|_{l^p_{\tau, N}W^{\sigma, q}} \leq C \|u^K(t_{k})\|_{l^p_{\tau, N}W^{\sigma, q}}.
    $$
    To estimate the remaining part, we just observe that
    $$
    \varphi_{1}(-2i \tau \Delta) ( 1 - \Pi_{\tau^{-{1 \over 2}}})u^K(t_{k}) =- {1 - \Pi_{\tau^{-{1 \over 2}}} \over 2 i \tau \Delta }   e^{2 i\tau \Delta} u^K(t_{k})  + {1 - \Pi_{\tau^{-{1 \over 2}}} \over 2 i \tau \Delta } u^K(t_{k}).
    $$
    Again, the multiplier  ${1 - \Pi_{\tau^{-{1 \over 2}}} \over 2 i \tau \Delta } $ is continuous on $L^q$ for every $q$ with norm uniform in $\tau.$, see \eqref{102} in Lemma \ref{lemfilter1}.
    $$
    \|\varphi_{1}(-2i \tau \Delta) ( 1 - \Pi_{\tau^{-{1 \over 2}}})u^K(t_{k})\|_{l^p_{\tau, N}W^{\sigma, q}}\leq C\left(  \|e^{2 i\tau \Delta} u^K(t_{k})\|_{l^p_{\tau, N}W^{\sigma, q}} + \|u^K(t_{k})\|_{l^p_{\tau, N}W^{\sigma, q}}   \right)
    $$
    and the result follows by using again Proposition~\ref{uKlpLq}.
  \end{proof}

  \begin{proposition}\label{DuKlpLq}
    Under the same assumption as in prop~\ref{uKlpLq}, we have the estimates
    \begin{equation}
       \sup_{s\in[-2\tau,2\tau]} \left\|e^{is\Delta}\nabla u^K(t_k)\right\|_{l^p_{\tau,N}L^q}
       \lesssim_\alpha \langle\varepsilon^\alpha N\tau\rangle^\frac1p (K\tau^\frac12)^\frac2p,
    \end{equation}
    for every \(\alpha>0\) (\(\alpha\leq4\) if \(d=1\)).
  \end{proposition}
  \begin{remark}
    We can by the same arguments to get 
    \begin{equation}
      \sup_{s\in[-2\tau,2\tau]} \left\|e^{is\Delta}u^K(t_k)\right\|_{l^p_{\tau,N}W^{1,q}}
      \lesssim_\alpha \langle\varepsilon^\alpha N\tau\rangle^\frac1p (K\tau^\frac12)^\frac2p.
    \end{equation}
  \end{remark}

  \begin{proof}
    By Duhamel's formula, we write
    \begin{equation}
      e^{is\Delta} \nabla u_K(t_n) = S_K(n\tau) (e^{is\Delta} \nabla u_0) 
      -i\varepsilon^2 \int_{0}^{\tau} \sum_{k=0}^{n-1} S_K(t_{n-k}+s-\tilde{s}) 
      \nabla \left(|\Pi_K u^K|^2 \Pi_K u^K\right)(t_k+\tilde{s}) \mathrm{d} \tilde{s}.
    \end{equation}
    We can use Strichartz estimate to get %TODO link to equation
    %\begin{multline}
    \[ \|e^{is\Delta} \nabla u_K(t_n)\|_{l^p_{\tau,N}L^q} \lesssim
      (K\tau^{\frac{1}{2}})^{\frac{2}{p}}\|e^{is\Delta}\nabla \phi\|_{L^2} 
      + \varepsilon^2
      (K\tau^{\frac{1}{2}})^{\frac{2}{p}}\sup_{\tilde{s}\in[0,\tau]}\left\| 
      \nabla \left(|\Pi_K u^K|^2 \Pi_K u^K\right)(t_k+\tilde{s}) \right\|_{l^1_{\tau,N}L^2} \]
    %\end{multline}
    By H\"older's inequality and Proposition~\ref{uKlpLq},
    \begin{align}
      \varepsilon^2 \sup_{\tilde{s}\in[0,\tau]}\left\| 
      \nabla \left(|\Pi_K u^K|^2 \Pi_K u^K\right)(t_k+\tilde{s}) \right\|_{l^1_{\tau,N}L^2} 
      &\leq \|\nabla u^K\|_{L^{\infty} L^2} \varepsilon^2 \sup_{\tilde{s} \in [0,\tau]} 
      \left\|\Pi_K u^K(t_k+\tilde{s})\right\|_{l_{\tau, N}^2 L^{\infty}}^2 \\
      & \lesssim_{\alpha}  {(\varepsilon^\alpha N\tau)}^\frac2\alpha 
      {\langle\varepsilon^\alpha N\tau\rangle}^{1-\frac2\alpha}. 
    \end{align}
    Consequently, we can find a suitable \(T_1\), such that for every \(
    \varepsilon^\alpha n \tau \leq T_1 \), we have 
    \[ 
      \sup_{s\in[-2\tau,2\tau]}\|e^{is\Delta}\nabla u^K(t_k)\|_{l^p_{\tau,N}L^q}
      \lesssim (K\tau^\frac12)^\frac2p.
    \]
    Hence, we can iterate the above estimate to get 
    \[ 
      \sup_{s\in[-2\tau,2\tau]}\|e^{is\Delta}\nabla u^K(t_k)\|_{l^p_{\tau,N}L^q}
      \lesssim \langle\varepsilon^\alpha N\tau\rangle^\frac1p (K\tau^\frac12)^\frac2p.
    \]
  \end{proof}

  \section{Error analysis}
  Following the discussing in~\cite{ORS21}, we set
  \begin{align}
    \label{TOp}& T(u)(\tau,t_n) = \int_0^{\tau} e^{-is\Delta} |u(t_n+s)|^2 u(t_n+s) \mathrm{d} s.\\
    \label{T1} & T_1(s, t_n) = -i\varepsilon^2 \left(e^{i s \Delta} \Pi_K u^K\left(t_n\right)\right)^2 e^{-i s \Delta} \Pi_K \bar{T}\left(\Pi_K u^K\right)\left(s, t_n\right) \\
    \label{T2} & T_2(s, t_n) = -2\varepsilon^4 \left(e^{i s \Delta} \Pi_K u^K\left(t_n\right)\right)\left|e^{i s \Delta} \Pi_K T\left(\Pi_K u^K\right)\left(s, t_n\right)\right|^2 \\
    \label{T3} & T_3(s, t_n) = i\varepsilon^6 \left|e^{i s \Delta} \Pi_K T\left(\Pi_K u^K\right)\left(s, t_n\right)\right|^2 e^{i s \Delta} \Pi_K T\left(\Pi_K u^K\right)\left(s, t_n\right) \\
    \label{T4} & T_4(s, t_n) = 2i\varepsilon^2 \left|e^{i s \Delta} \Pi_K u^K\left(t_n\right)\right|^2 e^{i s \Delta} \Pi_K T\left(\Pi_K u^K\right)\left(s, t_n\right) \\
    \label{T5} & T_5(s, t_n) = \varepsilon^4 \left(e^{-i s \Delta} \Pi_K \bar{u}^K\left(t_n\right)\right)\left(e^{i s \Delta} \Pi_K T\left(\Pi_K u^K\right)\left(s, t_n\right)\right)^2 \\
    \label{E1} & E_1(u^K, \tau, t_n) = i \int_0^{\tau} e^{-is\Delta} 
    [T_1 + T_2 + T_3 + T_4 + T_5](s, t_n) \mathrm{d} s,
  \end{align}
  We have the following error analysis
  \begin{multline}
    u^K(t_{n+1}) = e^{i\tau\Delta}u^K(t_n) 
    - i\varepsilon^2\tau S_K(\tau) \left( \left(\Pi_K u^K(t_n)\right)^2 
    \varphi_1(-2i\tau\Delta)\Pi_K \bar{u}^K(t_n) \right) \\
    + \varepsilon^2 S_K(\tau) \left( E_1(u^K, \tau, t_n) + E_2(u^K, \tau, t_n) 
    \right)
  \end{multline}
  where \(E_2\) is defined by 
  \begin{multline}\label{E2}
    E_2(u^K, \tau, t_n) = -2 \int_0^{\tau} \int_0^s e^{-is_1\Delta} \left[
    \nabla \left(e^{is_1\Delta}\Pi_K u^K(t_n)\right)^2 
    \nabla \left(e^{i(s_1-2s)\Delta}\overline{\Pi_K u^K(t_n)}\right)\right. \\
    \left.+ \left( \nabla e^{is_1\Delta}\Pi_K u^K(t_n) \right)^2 
    \left( e^{i(s_1-2s)\Delta}\overline{\Pi_K u^K(t_n)} \right) 
    \right] \mathrm{d}s_1\mathrm{d}s .
  \end{multline}
  
  Note that we can rewrite our scheme~\eqref{num-sol} in the form 
  \[ 
    u^{n+1} = e^{i\tau\Delta}u^n - i\varepsilon^2\tau S_K(\tau) 
    \left( (\Pi_K u^n)^2 \varphi_1(-2i\tau\Delta)\Pi_K \bar{u}^n \right).
  \]
  Let \( e^n = u^K(t_n) - u^n \) donote the error. Thus the error satisfies 
  the following 
  \begin{multline}
    e^{n+1} = e^{i\tau\Delta}e^n - i \varepsilon^2 \tau S_K(\tau) \left( 
    \left(\Pi_K u^K(t_n)\right)^2 \varphi_1(-2i\tau\Delta)\Pi_K \bar{u}^K(t_n)
    - (\Pi_K u^n)^2 \varphi_1(-2i\tau\Delta)\Pi_K \bar{u}^n \right) \\ 
    + \varepsilon^2 S_K(\tau) \left( E_1(u^K, \tau, t_n) + E_2(u^K, \tau, t_n) 
    \right),
  \end{multline}
  with \( e^0 = 0 \). Therefore, we obtain that 
  \begin{equation}
    e^n = -i \varepsilon^2 \tau \sum_{k=1}^{n-1}
    S_K(t_{n-k})G_k + \mathcal F_1^{n} + \mathcal F_1^{n},
  \end{equation}
  where
  \begin{align}
    \label{Gk} G_k = \left(\Pi_K u^K(t_n)\right)^2 \varphi_1(-2i\tau\Delta)\Pi_K 
    \bar{u}^K(t_n) - (\Pi_K u^n)^2 \varphi_1(-2i\tau\Delta)\Pi_K \bar{u}^n \\
    \label{F1F2} \mathcal F^{n}_1 = \varepsilon^2 \sum_{k=0}^{n-1} S_K(t_{n-k}) E_1(u^k, \tau, t_k), \quad
    \mathcal F^{n}_2 = \varepsilon^2 \sum_{k=0}^{n-1} S_K(t_{n-k}) E_2(u^k, \tau, t_k).
  \end{align}
  And we also have 
  \begin{equation}\label{emn}
    e^n = e^m -i \varepsilon^2 \tau \sum_{k=m}^{n-1}
    S_K(t_{n-k})G_k + 
    + \varepsilon^2 \sum_{k=m}^{n-1} S_K(t_{n-k}) E_1(u^k, \tau, t_k)
    + \varepsilon^2 \sum_{k=m}^{n-1} S_K(t_{n-k}) E_2(u^k, \tau, t_k)
  \end{equation}
  
  Next, we give the estimates of \(\mathcal F^n_1\) and \(\mathcal F^n_2\).
  \begin{lemma}\label{F1} %estimates of F_1
    For every \( T > 0 \), \( (p, q) \) admissible with \( p > 2 \) .
    We have the estimates for every \( \tau, \varepsilon
    \in (0,1] \) and every positive integer \(N\) satisfying \(\varepsilon^\alpha
    N\tau \leq T\)
    \begin{equation}
      \| \mathcal F_1^n \|_{l^p_{\tau,N}L^q} \lesssim_{\alpha,T}
      \varepsilon^2 (K\tau^\frac12)^\frac2p \tau 
      {(\varepsilon^\alpha N\tau)}^\frac2\alpha {\langle\varepsilon^\alpha N\tau\rangle}^{1-\frac2\alpha}.% , \quad
      % \| (\mathcal F_1{m,n})_{m \leq n \leq m+N} \|_{l^p_{\tau}W^{1,q}} \lessim_T
      % (K\tau^{\frac{1}{2}})^{\frac{2}{p}} \tau (N\tau\varepsilon^2).
    \end{equation}
  \end{lemma}

  \begin{remark}\label{remF1}
    By the same way of the proof we can also have the estimates 
    \[ \| S_K(-2\tau) \mathcal F_1^n \|_{l^p_\tau L^q} \lesssim_{\alpha,T}
    \varepsilon^2 (K\tau^\frac12)^\frac2p \tau 
    {(\varepsilon^\alpha N\tau)}^\frac2\alpha {\langle\varepsilon^\alpha N\tau\rangle}^{1-\frac2\alpha} \]
    and 
    \[ \left\| \left( \varepsilon^2 \sum_{k=m}^{n-1} S_K(t_{n-k}) E_1(u^k, \tau, t_k)
    \right)_{m \leq n \leq m+N} \right\|_{l^p_\tau L^q} \lesssim_{\alpha,T}
    \varepsilon^2 (K\tau^\frac12)^\frac2p \tau 
    {(\varepsilon^\alpha N\tau)}^\frac2\alpha {\langle\varepsilon^\alpha N\tau\rangle}^{1-\frac2\alpha}.% , \quad
    \]
  \end{remark}

  \begin{proof}
    First of all, by using the discrete Strichartz estimates
    \begin{equation}\label{F1lplq}
      \begin{aligned}
        \|\mathcal F_1^n\|_{l^p_{\tau,N}L^q} & \lesssim \varepsilon^2 (K\tau^\frac12)^\frac2p
        \tau^{-1} \|E_1(u^K, \tau, t_k)\|_{l^1_{\tau,N}L^2} \\ 
        &\lesssim \varepsilon^2 (K\tau^\frac12)^\frac2p \sup_{s \in [0,\tau]}
        \left(\sum_{i=1}^5 \|T_i(t_n,s)\|_{l^1_{\tau,N}L^2} \right).
      \end{aligned}
    \end{equation}
    Next by using~\eqref{T1} and Sobolev embedding, we have 
    \begin{equation}
      \begin{aligned}
        \|T_1(t_n,s)\|_{L^2} &\lesssim \varepsilon^2\|e^{is\Delta}u^K\|^2_{L^6} \|e^{is\Delta} T(\Pi_K u^K)(s,t_n)\|_{L^6} \\ 
        &\lesssim \varepsilon^2\|e^{is\Delta}u^K\|^2_{H^1} \|e^{is\Delta} T(\Pi_K u^K)(s,t_n)\|_{H^1} \\ 
        &\lesssim \varepsilon^2\|u^K\|^2_{H^1} \|T(\Pi_K u^K)(s,t_n)\|_{H^1} \\ 
      \end{aligned}
    \end{equation}
    Next, by using~\eqref{TOp},
    \begin{equation}\label{TH1}
      \begin{aligned}
        & \|T(\Pi_K u^K)(s,t_n)\|_{H^1} \\ & \lesssim 
        \int_0^\tau \left( 
        \left\| |\Pi_K u^K(t_n+\tilde{s})|^2 \nabla\Pi_K u^K(t_n+\tilde{s}) \right\|_{L^2}
        \left\| |\Pi_K u^K(t_n+\tilde{s})|^2 \Pi_K u^K(t_n+\tilde{s}) \right\|_{L^2}
        \right)\,\mathrm{d}\tilde{s} \\ 
        & \lesssim \|u^K\|_{L^\infty H^1} \int_0^\tau
        \|u^K(t_n+\tilde{s})\|_{L^\infty}^2\,\mathrm{d}\tilde{s}.
      \end{aligned}
    \end{equation}
    Hence, 
    \begin{equation}\label{Tl1H1}
      \|T(\Pi_K u^K)(s,t_n)\|_{l^1 H^1} \lesssim \tau \|u^K\|_{L^\infty H^1}
      \sup_{\tilde{s} \in [0,\tau]} \|u^K(t_n+\tilde{s})\|^2_{l^2_{\tau,N}L^\infty}.
    \end{equation}
    We thus obtain
    \[ \|T_1(t_n,s)\|_{l^1_{\tau,N}L^2} \lesssim \varepsilon^2\|u^K\|_{L^\infty H^1}^2 \|T(\Pi_K u^K)(s,t_n)\|_{l^1 H^1}
    \lesssim \tau \|u^K\|_{L^\infty H^1}^3 \varepsilon^2\sup_{\tilde{s} \in [0,\tau]} 
    \|u^K(t_n+\tilde{s})\|^2_{l^2_{\tau,N}L^\infty} .\]
    By using~\eqref{uKl2Linfty}, we finally obtain 
    \begin{equation}\label{T1l1L2}
      \|T_1(t_n,s)\|_{l^1_{\tau,N}L^2} \lesssim \tau 
      {(\varepsilon^\alpha N\tau)}^\frac2\alpha {\langle\varepsilon^\alpha N\tau\rangle}^{1-\frac2\alpha}.
    \end{equation}
    In a similar way, we obtain that 
    \begin{align*}
      \|T_2(s,t_n)\|_{L^2} & \lesssim  \varepsilon^4 \|e^{is\Delta} u^K(t_n)\|_{L^6}
      \| e^{is\Delta}T(\Pi_K u^K)(s,t_n) \|_{L^6}^2 \\
      &\lesssim \varepsilon^4 \|u^K(t_n)\|_{H^1} \|T(\Pi_K u^K)(s, t_n) \|_{H^1}^2
    \end{align*}
    Hence, we get 
    \[ \|T_2(s,t_n)\|_{l^1_{\tau,N}L^2} \lesssim \|u\|_{L^\infty_{T}H^1} \varepsilon^2 
    \|T(\Pi_K u^K)(s,t_n)\|_{l^\infty_{\tau,N}H^1} \varepsilon^2 
    \|T(\Pi_K u^K)(s,t_n)\|_{l^1_{\tau,N}H^1}.\]
    By using~\eqref{Tl1H1} to estimate \(\varepsilon^2\|T(\Pi_K u^K)(s,t_n)\|_{l^1 H^1}\), we
    only need to estimate the last term. By using again~\eqref{TH1}, we obtain that
    \[ \|T(\Pi_K u^K)(s,t_n)\|_{l^\infty_{\tau,N}H^1} \lesssim \int_0^\tau \|u^K(t_{n}+s)\|_{H^1} \|u^K(t_n+ s) \|_{L^\infty}^2\,\mathrm{d}s \lesssim \|u^K\|_{L^\infty_{T}H^1} \|u^K\|_{L^2_{T}L^\infty}^2 \]
    and, therefore by~\eqref{uL2Linfty},
    \begin{equation}\label{TlinftyH1}
      \varepsilon^2\|t(\pi_k u^k)(s,t_n)\|_{l^\infty_{\tau,n}h^1} \leq 
      \varepsilon^2 \|u^k\|_{l^\infty h^1} \|u^k\|_{l^2_{n\tau} l^\infty}^2 
      \leq {(\varepsilon^\alpha n\tau)}^\frac2\alpha {\langle\varepsilon^\alpha n\tau\rangle}^{1-\frac2\alpha} 
      \lesssim_T 1,
    \end{equation}
    for all \(N\) satisfying \( \varepsilon^\alpha N\tau \leq T \).
    We thus finally obtain that 
    \begin{equation}\label{T2l1L2}
      \|T_2(t_n,s)\|_{l^1_{\tau,N}L^2} \lesssim_T \tau 
      {(\varepsilon^\alpha N\tau)}^\frac2\alpha {\langle\varepsilon^\alpha N\tau\rangle}^{1-\frac2\alpha}.
    \end{equation}
    Finally, from the same arguments as above, we have that 
    \[ \|T_3(s,t_n)\|_{L^2}  \lesssim \left(\varepsilon^2\|T(\Pi_K u^K)(s, t_n) \|_{H^1}\right)^3 \]
    and thus 
    \[ \|T_3(t_n,s)\|_{l^1_{\tau,N}L^2} \lesssim \left(\varepsilon^2 
    \|T(\Pi_K u^K)(s,t_n)\|_{l^\infty_{\tau,N}H^1}\right)^2 \varepsilon^2 
    \|T(\Pi_K u^K)(s,t_n)\|_{l^1_{\tau,N}H^1}. \]
    Therefore, by using~\eqref{Tl1H1} and~\eqref{TlinftyH1}, we obtain that 
    \begin{equation}\label{T3l1L2}
      \|T_3(t_n,s)\|_{l^1_{\tau,N}L^2} \lesssim_T \tau 
      {(\varepsilon^\alpha N\tau)}^\frac2\alpha {\langle\varepsilon^\alpha N\tau\rangle}^{1-\frac2\alpha}.
    \end{equation}
    The term \(T_4\) is estimated in the same way as \(T_1\), the term \(T_5\)
    in the same way as \(T_2\). Finally, by combining~\eqref{F1lplq},~\eqref{T1l1L2},~\eqref{T2l1L2} 
    and~\eqref{T3l1L2}, we obtain that 
    \[ \| \mathcal F_1^n \|_{l^p_{\tau,N}L^q} \lesssim_{\alpha,T}
    \varepsilon^2 (K\tau^\frac12)^\frac2p \tau 
    {(\varepsilon^\alpha N\tau)}^\frac2\alpha {\langle\varepsilon^\alpha N\tau\rangle}^{1-\frac2\alpha}
    \]
  \end{proof}

  \begin{lemma}\label{F2} %estimates of F_2
    Under the same assumption of the previous lemma, we have 
    \begin{align}
      &\label{F2d1p} \| \mathcal F_2^n \|_{l^p_{\tau,N}L^q} \lesssim (K\tau^\frac12)^{\frac2p+\frac12} \tau 
      {(\varepsilon^4 N\tau)}^\frac12 {\langle\varepsilon^4 N\tau\rangle}^\frac12 ,\quad \text{if }d=1, \\
      &\label{F2d2p} \| \mathcal F_2^n \|_{l^p_{\tau,N}L^q} \lesssim_\alpha \varepsilon (K\tau^\frac12)^{\frac2p+1} \tau 
      {(\varepsilon^\alpha N\tau)}^\frac1\alpha {\langle\varepsilon^\alpha N\tau\rangle}^{1-\frac1\alpha} ,\quad \text{if }d=2, \\
      &\label{F2d3} \| \mathcal F_2^n \|_{l^p_{\tau,N}L^q} \lesssim_\alpha \varepsilon^2 (K\tau^\frac12)^{\frac2p+2} \tau \log{K}^\frac23
      \langle\varepsilon^\alpha N\tau\rangle ,\quad \text{if }d=3.
    \end{align}
    Moreover, by~\eqref{F2d1p} and~\eqref{F2d2p}, we cound have for every \( 0 <
    \eta < 1\)
    \begin{align}
      &\label{F2d1} \| \mathcal F_2^n \|_{l^p_{\tau,N}L^q} \lesssim_\alpha \varepsilon^{2-\frac\alpha2} (K\tau^\frac12)^{\frac2p+\frac12} \tau 
      {(\varepsilon^\alpha N\tau)}^\frac12 {\langle\varepsilon^\alpha N\tau\rangle}^\frac12 ,\quad \text{if }d=1,\alpha \leq 4 \\
      &\label{F2d2} \| \mathcal F_2^n \|_{l^p_{\tau,N}L^q} \lesssim_{\alpha,\eta} \varepsilon^{1+\eta} (K\tau^\frac12)^{\frac2p+1} \tau 
      {(\varepsilon^\alpha N\tau)}^\frac{1-\eta}\alpha {\langle\varepsilon^\alpha N\tau\rangle}^{1-\frac{1-\eta}\alpha} ,\quad \text{if }d=2
    \end{align}
  \end{lemma}

  \begin{remark}\label{remF2}
    As in the previous lemma, we can also have the corresponding estimates for
    \( \|S_K(-2\tau)\mathcal F^2_n\|_{l^p_{\tau,N}L^q} \) and the sequence 
    \( (\varepsilon^2 \sum_{k=m}^{n-1} S_K(t_{n-k}) E_2(u^k, \tau, t_k))_{m \leq n \leq m+N} \).
  \end{remark}

  \begin{proof}
    At first, we observe that using the expressions~\eqref{E2},~\eqref{F1F2}, we can write that
    \begin{equation}%\label{1F2n0}
      \begin{aligned}
        \mathcal F_2^n &= 2\varepsilon^2 \int_0^\tau \int_0^s e^{-is_1\Delta}
        \sum_{k=0}^{n-1} S_K(t_{n-k})  G(s,s_1,t_k)\,\mathrm{d}s_1\,\mathrm{d}s \\
        & = 2\varepsilon^2 \int_0^\tau \int_0^s \sum_{k=0}^{n-1}  S_K(t_{n-k}- s_1)
        G(s,s_1,t_k)\,\mathrm{d}s_1\,\mathrm{d}s,
      \end{aligned}
    \end{equation}
    where
    \begin{multline*}
      G(s,s_1,t_k) =  -\nabla\left( e^{is_1\Delta} \Pi_K u^K(t_k) \right)^2 
      \nabla \left( e^{i(s_1-2s)\Delta} \overline{\Pi_K u^K(t_k)} \right) \\
      + \left( \nabla e^{is_1\Delta} \Pi_K u^K(t_k) \right)^2 \left( e^{i(s_1-2s)\Delta} \overline{\Pi_K u(t_k)} \right)
    \end{multline*}
    and we observe that \(s/\tau\), \(s_1/\tau\), \((s_1- 2s)/\tau\) are uniformly
    bounded in \([-2, 1]\) so that we will be able to use~Theorem~\ref{thmDSE}
    and Proposition~\ref{uKlpLq} and~\ref{DuKlpLq}.
    We first estimate
    \begin{equation}%\label{1F2n}
      \|\mathcal F_2^n\|_{l^p_{\tau,N}L^q} \lesssim \varepsilon^2\tau^2  
      \sup_{0 \leq s_1 \leq s \leq \tau} \left\|\sum_{k=0}^{n-1}  S_K(t_{n-k}-s_1)
      G(s, s_1, t_k) \right\|_{l^p_{\tau, N}L^q}.
    \end{equation}
    Then, using  discrete Strichartz estimates, we obtain that
    \[ \| \mathcal F_{2}^n \|_{l^p_{\tau, N}L^{q}} \lesssim \varepsilon^2\tau
    (K \tau^\frac12)^\frac2p \sup_{0 \leq s, s_1 \leq \tau}\ \|G(s, s_1,
    t_k) \|_{l^1_{\tau, N} L^2}. 
    \]
    We shall then use slightly different arguments depending on the dimension.
    In dimension \(d \leq 2\), we use H\"older's inequality to get
    \begin{multline*}
      \|G(s,s_1,t_k)\|_{L^2} \lesssim  \|\nabla e^{-i(s_1-2s)\Delta}\Pi_{K}u^K(t_k)\|_{L^4} 
      \|\nabla e^{is_1\Delta} \Pi_{K}u^K(t_k)\|_{L^4}
      \|e^{is_1\Delta}\Pi_{K}u^K(t_k)\|_{L^\infty}  \\
      + \|\nabla e^{is_1\Delta}\Pi_{K}u^K(t_k)\|_{L^4}^2 \|e^{-i(s_1-2s)\Delta}\Pi_{K}u^K(t_k)\|_{L^\infty}
    \end{multline*}
    and therefore,
    \begin{equation}\label{F2lpLq}
      \|\mathcal F_2^n |_{l^p_{\tau, N}L^q} \lesssim \varepsilon^2 \tau 
      (K\tau^\frac12)^\frac2p \left(\sup_{\hat{s} \in [-2\tau,\tau]}  
      \|\nabla e^{i\hat{s}\Delta}\Pi_K u^K(t_k)\|_{l^4_{\tau, N}L^4}  \right)^2 
      \sup_{\hat s \in [-2\tau, \tau]} \|e^{i\hat{s}\Delta}\Pi_K u^K(t_k)\|_{l^2_{\tau, N}L^\infty}.
    \end{equation}
    Next, we use the estimate
    \begin{equation}\label{expuKl2Linfty}
      \sup_{\hat s \in [-2\tau, 2\tau]} \|e^{i \hat s\Delta }\Pi_{K} u^K(t_{k})\|_{l^2_{\tau, N}L^\infty} 
      \lesssim_{\alpha} {(N\tau)}^\frac1\alpha {\langle\varepsilon^\alpha N\tau\rangle}^{\frac12-\frac1\alpha},
    \end{equation}
    from~\eqref{uKl2Linfty}. %Indeed, as noticed after Proposition \ref{propuKinfty}, in \black dimensions $1$ and 2, \black this estimate is true without further restriction on $\alpha \geq 1$. \black Moreover, for all $\hat s \in [-2\tau, 2\tau]$, \black we have the estimate
    For \(d=1\), using H\"older and~\eqref{DuKlpLq}, we have
    \begin{equation}\label{1Dul4L4}
      \sup_{\hat{s} \in [-2\tau,\tau]} \|\nabla e^{i\hat{s}\Delta}\Pi_K u^K(t_k)\|_{l^4_{\tau, N}L^4}
    \leq (N\tau)^\frac18 \sup_{\hat{s} \in [-2\tau,\tau]} \|\nabla e^{i\hat{s}\Delta}\Pi_K u^K(t_k)\|_{l^8_{\tau, N}L^4}
    \lesssim_\alpha (N\tau)^\frac18 \langle\varepsilon^4 N\tau\rangle^\frac18 (K\tau^\frac12)^\frac14.
    \end{equation}
    while for \(d=2\), since \((4,4)\) is admissible, we get 
    \begin{equation}\label{2Dul4L4}
      \sup_{\hat{s} \in [-2\tau,\tau]} \|\nabla e^{i\hat{s}\Delta}\Pi_K u^K(t_k)\|_{l^4_{\tau, N}L^4}
    \lesssim_\alpha \langle\varepsilon^\alpha N\tau\rangle^\frac14 (K\tau^\frac12)^\frac12.
    \end{equation}
    Consequently, by combining~\eqref{F2lpLq},~\eqref{expuKl2Linfty} and~\eqref{1Dul4L4},
    we get the desired estimate for \(d=1\)
    \[ \| \mathcal F_2^n \|_{l^p_{\tau,N}L^q} \lesssim_T (K\tau^\frac12)^{\frac2p+\frac12} \tau 
    {(\varepsilon^4 N\tau)}^\frac12 {\langle\varepsilon^4
    N\tau\rangle}^\frac12 , \]
    and by combining~\eqref{F2lpLq},~\eqref{expuKl2Linfty} and~\eqref{2Dul4L4},
    we get for \(d=2\)
    \[ \| \mathcal F_2^n \|_{l^p_{\tau,N}L^q} \lesssim \varepsilon (K\tau^\frac12)^{\frac2p+1} \tau 
    {(\varepsilon^\alpha N\tau)}^\frac1\alpha {\langle\varepsilon^\alpha N\tau\rangle}^{1-\frac1\alpha} .\]
    In dimension \(3\), since \((4,4)\) is not admissible. We use in place the
    estimate 
    \begin{equation}
      \| \mathcal F_2^n \|_{l^p_{\tau, N}L^q} \lesssim \varepsilon^2 \tau 
      (K\tau^\frac12)^\frac2p \left(\sup_{\hat s \in [-2\tau, \tau]} 
      \|\nabla e^{i\hat s\Delta}\Pi_{K}u^K(t_k)\|_{l^\frac83_{\tau,N}L^4} \right)^2 
      \sup_{\hat s \in [-2\tau, \tau]} \|e^{i\hat s\Delta }\Pi_K u^K(t_k)\|_{l^4_{\tau, N}L^\infty}
    \end{equation}
    and therefore, we get from Proposition~\ref{DuKlpLq} that 
    \[ \| \mathcal F_2^n \|_{l^p_{\tau, N}L^q} \lesssim \varepsilon^2 \tau 
    (K\tau^\frac12)^{\frac2p+\frac32} {\langle\varepsilon^\alpha N\tau\rangle}^\frac34 
    \sup_{\hat s \in [-2\tau, \tau]} \|e^{i\hat s\Delta }\Pi_K u^K(t_k)\|_{l^4_{\tau, N}L^\infty} \]
    Here we cannot use anymore Proposition~\ref{uKlpLq} in order to estimate 
    \(\|e^{i \hat s\Delta }\Pi_{K} u^K(t_{k})\|_{l^4_{\tau, N}L^\infty}\). We 
    are thus forced to use Sobolev embedding and proposition~\ref{DuKlpLq}. 
    Thanks to Lemma~\ref{sobbord}, 
    \[ \|e^{i\hat s\Delta }\Pi_K u^K(t_k)\|_{l^4_{\tau, N}L^\infty} \lesssim 
    (\log K)^\frac23 \|e^{i\hat s\Delta }\Pi_K u^K(t_k)\|_{l^4_{\tau, N}W^{1,3}}
    \lesssim (\log K)^\frac23 {\langle\varepsilon^\alpha N\tau\rangle}^\frac14 
    (K\tau^\frac12)^\frac12 .\]
    This finally yields 
    \[ \| \mathcal F_2^n \|_{l^p_{\tau,N}L^q} \lesssim \varepsilon^2 (K\tau^\frac12)^{\frac2p+2}
    \tau \log{K}^\frac23 \langle\varepsilon^\alpha N\tau\rangle.\]

    Finally, estimate~\eqref{F2d1} is directly get from~\eqref{F2d1p} by using
    \(\alpha\leq4\). And by substituting \(\alpha\) by \(\frac\alpha{1-\eta}\) 
    in~\eqref{F2d2p} for some \(0<\eta<1\) and using 
    \(\varepsilon^\frac\alpha{1-\eta}\leq\varepsilon^\alpha\), then we
    get~\eqref{F2d2}.
  \end{proof}

  \section{Proof of the main theorem}
  At first, we use Corollary~\ref{coru-uKLinftyL2} to write that 
  \[ \| u(t_n) - u^n \|_{L^2} \leq \|u(t_n) - u^K(t_n)\|_{L^2} + \|u^K(t_n) - u^n\|_{L^2}
  \lesssim_{\alpha,T} \frac1K + \|e^n\|_{L^2}, \]
  for every \( 0 \leq \varepsilon^\alpha n\tau \leq T.\)
  We shall choose the parameter \(K = \varepsilon^{-\beta} \tau^{-\frac\gamma2}\)
  to get the optimal error. The choice will depend on the dimension. 
  By using Lemma~\ref{F1} and Lemma~\ref{F2}, we
  have that for every \(0<\eta<1\)
  \begin{equation*}
    \begin{aligned}
      & \|\mathcal F_1^n\|_{l^\infty_{\tau,N}L^2} + 
      \|\mathcal F_2^n\|_{l^\infty_{\tau,N}L^2} \lesssim_{T} \varepsilon^2 \tau +
      \varepsilon^{2-\frac\alpha2}\tau (K\tau^\frac12)^\frac12,\quad \text{if } d=1, \\
      & \|\mathcal F_1^n\|_{l^\infty_{\tau,N}L^2} + 
      \|\mathcal F_2^n\|_{l^\infty_{\tau,N}L^2} \lesssim_{T} \varepsilon^2 \tau +
      \varepsilon^{1+\eta} \tau (K\tau^\frac12),\quad \text{if } d=2, \\
      & \|\mathcal F_1^n\|_{l^\infty_{\tau,N}L^2} + 
      \|\mathcal F_2^n\|_{l^\infty_{\tau,N}L^2} \lesssim_{T} \varepsilon^2 \tau +
      \varepsilon^2 \tau (K\tau^\frac12)^2 \log{K}^\frac23,\quad \text{if } d=3.
    \end{aligned}
  \end{equation*}
  We thus choose \(\beta,\gamma\) such that \(\|\mathcal F_2^n\|_{l^\infty_{\tau,N}L^2}\) 
  have the same order as \(\frac1K\) which gives
  \begin{equation}\label{bg}
    \begin{aligned}
      & \beta = \frac{4-\alpha}3, \gamma = \frac53 ,\quad \text{if } d=1, \\ 
      & \beta = \frac{1+\eta}2, \gamma = \frac32, \quad \text{if } d=2, \\ 
      & \beta = \frac23, \gamma = \frac43, \quad \text{if } d=3. 
    \end{aligned}
  \end{equation}
  For simplicity, we write 
  \begin{equation}\label{delta}
    \begin{aligned}
      \delta &= \frac1K = \varepsilon^\beta \tau^{\frac\gamma2}, \quad \text{if } d=1,2\\ 
      \delta &= \varepsilon^\beta \tau^{\frac\gamma2}(|\log\tau|+|\log\varepsilon|)^\frac23,
      \quad \text{if } d=3 
    \end{aligned}
  \end{equation}
  Under this CFL type condition, we get that 
  \begin{equation}\label{F1F2linftyL2}
      \|\mathcal F_1^n\|_{l^\infty_{\tau,N}L^2} + 
      \|\mathcal F_2^n\|_{l^\infty_{\tau,N}L^2} \lesssim_T \delta
  \end{equation}
  and more generally 
  \begin{equation}\label{F1F2lpLq}
      \|\mathcal F_1^n\|_{l^p_{\tau,N}L^q} + 
      \|\mathcal F_2^n\|_{l^p_{\tau,N}L^q} \lesssim_T \delta
      (K\tau^\frac12)^\frac2p,
  \end{equation}
  %For simplicity, we define a small \(T_1 \leq T\) and assume that all the index 
  %\(k\) below satisfies \(\varepsilon^\alpha k\tau leq T_1\). We will prove by
  %induction that \(\|e^n\|_{l^\infty L^2}\|\) is bounded by some constant
  %depending on \(T\) for sufficiently small \(T_1\) and \(\tau\).

  We define the normalized norm 
  \begin{equation}\label{Xtau}
      \|e^n\|_{X_{\tau,k}} := \frac1\delta \|e^n\|_{l^\infty_{\tau,k}L^2} 
      + \frac1{\delta(K\tau^\frac12)^\frac d4} \left(\|e^n\|_{l^\frac8d_{\tau,k}L^4}
      + \|S_K(-2\tau)e^n\|_{l^\frac8d_{\tau,k}L^4}\right).
  \end{equation}
  Note that by substituting \(u^k = u^K(t_k) - e^k \), we can write~\eqref{Gk}
  as 
  \[ G_k = G_k^1 + G_k^2 + G_k^3, \]
  where 
  \begin{equation*}
    \begin{aligned}
      G_k^1 &= 2(\Pi_K u^K(t_k))(\varphi_1(-2i\tau\Delta)\Pi_K\overline u^K(t_k))(\Pi_K e^k)
      + (\Pi_K u^K(t_k))^2\varphi_1(-2i\tau\Delta)\Pi_K\overline e^k, \\
      G_k^2 &= -\left(\varphi_1(-2i\tau\Delta)\Pi_K \overline u^K(t_k)\right)(\Pi_K e^k)^2 
      - 2(\Pi_K u^K(t_k))(\Pi_k e^k)\varphi_1(-2i\tau\Delta)\Pi_K \overline e^k, \\
      G_k^3 &= (\Pi_K e^k)^2\varphi_1(-2i\tau\Delta)\Pi_K\overline e^k.
    \end{aligned}
  \end{equation*}
  To estimates \(e^n\), we use the discrete Strichartz estimates of
  Theorem~\ref{thmDSE}, we get that 
  \begin{equation}\label{enlinftyL2}
    \|e^n\|_{l^\infty_{\tau,k+1}L^2} \lesssim \|\mathcal F_1^n\|_{l^\infty_{\tau,k}L^2} + 
    \|\mathcal F_2^n\|_{l^\infty_{\tau,k}L^2} + \varepsilon^2\|G_n^1\|_{l^1_{\tau,k}L^2} 
    + \varepsilon^2\|G_n^2\|_{l^1_{\tau,k}L^2} + (K\tau^\frac12)^\frac d4 
    \varepsilon^2\|G_n^3\|_{l^{(\frac8d)'}_{\tau,k}L^\frac43} .
  \end{equation}
  The first two terms are bounded by~\eqref{F1F2linftyL2}. To estimate the third
  term, we use that 
  \[ \varepsilon^2\|G_n^1\|_{l^1_{\tau,k}L^2} \lesssim \left(\|e^n\|_{l^\infty_{\tau,k}L^2}
  + \|\varphi_1(-2i\tau\Delta)e^n\|_{l^\infty_{\tau,k}L^2} \right)  \left( 
  \varepsilon^2 \|u^K(t_n)\|_{l^2_{\tau,k}L^\infty}^2 + 
  \varepsilon^2 \|\varphi_1(-2i\tau\Delta)\Pi_K u^K(t_n)\|_{l^2_{\tau,k}L^\infty}^2 \right). \]
  By using~\eqref{uKl2Linfty} and~\eqref{phiuKlpLq}, we can estimate the right
  hand side by 
  \[ \varepsilon^2 \|u^K(t_n)\|_{l^2_{\tau,k}L^\infty}^2 + \varepsilon^2 
  \|\varphi_1(-2i\tau\Delta)\Pi_K u^K(t_n)\|_{l^2_{\tau,k}L^\infty}^2 \lesssim 
  {(\varepsilon^\alpha k\tau)}^\frac2\alpha {\langle\varepsilon^\alpha k\tau\rangle}^{1-\frac2\alpha}.\]
  Combining with~\eqref{103}, we get that
  \begin{equation}\label{G1}
    \varepsilon^2\|G_n^1\|_{l^1_{\tau,k}L^2} \lesssim \|e^n\|_{l^\infty_{\tau,k}L^2}
    {(\varepsilon^\alpha k\tau)}^\frac2\alpha {\langle\varepsilon^\alpha k\tau\rangle}^{1-\frac2\alpha}.
  \end{equation}
  Let's now estimate \(G_n^2\). From the same argument, we obtain that 
  \[ \varepsilon^2\|G_n^2\|_{l^1_{\tau,k}L^2} \lesssim  \varepsilon
  \left(\|e^n\|^2_{l^4_{\tau,k}L^4} + \|\varphi_1(-2i\tau\Delta)\Pi_K e^n\|^2_{l^4_{\tau,k}L^4} \right)
  \left( \varepsilon \|u^K(t_n)\|_{l^2_{\tau,k}L^\infty} + 
  \varepsilon \|\varphi_1(-2i\tau\Delta)\Pi_K u^K(t_n)\|_{l^2_{\tau,k}L^\infty} \right). \]
  which yields 
  \[ \varepsilon^2\|G_n^2\|_{l^1_{\tau,k}L^2} \lesssim  \varepsilon
  \left(\|e^n\|^2_{l^4_{\tau,k}L^4} + \|\varphi_1(-2i\tau\Delta)\Pi_K e^n\|^2_{l^4_{\tau,k}L^4} \right)
  {(\varepsilon^\alpha k\tau)}^\frac1\alpha {\langle\varepsilon^\alpha k\tau\rangle}^{\frac12-\frac1\alpha}. \]
  We will estimate the \(l^4L^4\) norm depending on dimension. Note that pair 
  \((\frac8d,4)\) is admissible. For \(d=1\), note that in this situation we
  assume that \(\alpha \leq 4\), hence by h\"older, we have 
  \begin{equation*}
    \begin{aligned}
      \varepsilon \left(\|e^n\|^2_{l^4_{\tau,k}L^4} + \|\varphi_1(-2i\tau\Delta)\Pi_K e^n\|^2_{l^4_{\tau,k}L^4} \right)
      &\leq \varepsilon (k\tau)^{\frac12-\frac d4} \left(\|e^n\|^2_{l^\frac8d_{\tau,k}L^4} 
      + \|\varphi_1(-2i\tau\Delta)\Pi_K e^n\|^2_{l^\frac8d_{\tau,k}L^4} \right) \\
      &\leq  (\varepsilon^\alpha k\tau)^{\frac12-\frac d4}
      \left(\|e^n\|^2_{l^\frac8d_{\tau,k}L^4} 
      + \|\varphi_1(-2i\tau\Delta)\Pi_K e^n\|^2_{l^\frac8d_{\tau,k}L^4} \right). 
    \end{aligned}
  \end{equation*}
  For \(d=2\), \((4,4)\) is exactly \((\frac8d,4)\) which is admissible. And for
  \(d=3\), since \((\frac83,4) < 4\) we will use the reverse inclusion rule for
  the discrete \(l^p_\tau\) spaces, which is 
  \[ \|f\|_{l^p_\tau X} \leq \frac1{\tau^{\frac1q-\frac1p}} \|f\|_{l^q_\tau X}, 
  \quad p > q.\]
  Thus 
  \[ \varepsilon \left(\|e^n\|^2_{l^4_{\tau,k}L^4} + \|\varphi_1(-2i\tau\Delta)\Pi_K e^n\|^2_{l^4_{\tau,k}L^4} \right)
  \leq \varepsilon \tau^{\frac12-\frac d4} \left(\|e^n\|^2_{l^\frac8d_{\tau,k}L^4} + 
  \|\varphi_1(-2i\tau\Delta)\Pi_K e^n\|^2_{l^\frac8d_{\tau,k}L^4}\right). \]
  As we use in the proof of Corollary~\ref{corphiuKlpLq}, we use the
  decomposition
  \[ \varphi_1(-2i\tau\Delta)\Pi_K e^n = \varphi_1(-2i\tau\Delta)\Pi_{\tau^{-\frac12}}\Pi_K e^n
  - \frac{1 - \Pi_{\tau^{-\frac12}}}{2i\tau\Delta} e^{-2i\tau\Delta}\Pi_K e^n
  + \frac{1 - \Pi_{\tau^{-\frac12}}}{2i\tau\Delta}\Pi_K e^n. \]
  We have that 
  \begin{equation}\label{phie}
    \|\varphi_1(-2i\tau\Delta)\Pi_K e^n\|_{L^4} 
    \lesssim \|S_K(-2\tau)e^n\|_{L^4} + 
    \|e^n\|_{L^4}.
  \end{equation}
  For summary, we have 
  \begin{equation}\label{G2}
    \begin{aligned}
      \varepsilon^2\|G_n^2\|_{l^1_{\tau,k}L^2} &\lesssim (\varepsilon^\alpha k\tau)^{\frac14+\frac1\alpha}
      \langle\varepsilon^\alpha k\tau\rangle^{\frac12-\frac1\alpha}
      \left(\|e^n\|^2_{l^\frac8d_{\tau,k}L^4} + \|S_K(-2\tau)e^n\|_{l^\frac8d_{\tau,k}L^4}^2\right),\quad \text{if }d=1,\\
      \varepsilon^2\|G_n^2\|_{l^1_{\tau,k}L^2} &\lesssim \varepsilon
      (\varepsilon^\alpha k\tau)^\frac1\alpha \langle\varepsilon^\alpha k\tau\rangle^{\frac12-\frac1\alpha}
      \left(\|e^n\|^2_{l^\frac8d_{\tau,k}L^4} + \|S_K(-2\tau)e^n\|_{l^\frac8d_{\tau,k}L^4}^2\right),\quad \text{if }d=2,\\
      \varepsilon^2\|G_n^2\|_{l^1_{\tau,k}L^2} &\lesssim \varepsilon \tau^{-\frac14}
      (\varepsilon^\alpha k\tau)^\frac1\alpha \langle\varepsilon^\alpha k\tau\rangle^{\frac12-\frac1\alpha}
      \left(\|e^n\|^2_{l^\frac8d_{\tau,k}L^4} + \|S_K(-2\tau)e^n\|_{l^\frac8d_{\tau,k}L^4}^2\right),\quad \text{if }d=3,\\
    \end{aligned}
  \end{equation}
  Finally, to estimate the last term of~\eqref{enlinftyL2}, by H\"older
  and~\eqref{phie},
  \[ \|G_n^3\|_{l^{(\frac8d)'}_{\tau,k}L^\frac43} \lesssim \left(
  \left\| \|e^n\|_{L^4}^3 \right\|_{l^{(\frac8d)'}_{\tau,k}} 
  + \left\| \|S_K(-2\tau)e^n \|_{L^4}^3 \right\|_{l^{(\frac8d)'}_{\tau,k}}\right). \]
  For \(d=1\), again use \(\alpha\leq4\), we obtain by H\"older that 
  \begin{equation}\label{G3d1}
    \varepsilon^2 \|G_n^3\|_{l^{(\frac8d)'}_{\tau,k}L^\frac43} \lesssim 
    (\varepsilon^\alpha k\tau)^\frac12 \left(
    \|e^n\|^3_{l^\frac8d_{\tau,k}L^4} + \|S_K(-2\tau)e^n\|_{l^\frac8d_{\tau,k}L^4}^3\right).
  \end{equation}
  For \(d=2\), since \((\frac8d)' = \frac13\frac8d\), we obtain again by
  H\"older that 
  \begin{equation}\label{G3d2}
    \varepsilon^2 \|G_n^3\|_{l^{(\frac8d)'}_{\tau,k}L^\frac43} \lesssim \varepsilon^2
    \left(\|e^n\|^3_{l^\frac8d_{\tau,k}L^4} + \|S_K(-2\tau)e^n\|_{l^\frac8d_{\tau,k}L^4}^3\right).
  \end{equation}
  For \(d=3\), since \((\frac8d)' > \frac13\frac8d\), we use again by reverse
  inclusion of \(l^p_\tau\) and \( \frac1{(\frac8d)'} - \frac{8d}3 = -\frac12\)
  to get that 
  \begin{equation}\label{G3d3}
    \varepsilon^2 \|G_n^3\|_{l^{(\frac8d)'}_{\tau,k}L^\frac43} \lesssim \varepsilon^2 \tau^{-\frac12}
    \left(\|e^n\|^3_{l^\frac8d_{\tau,k}L^4} + \|S_K(-2\tau)e^n\|_{l^\frac8d_{\tau,k}L^4}^3\right).
  \end{equation}

  Next, we need to estimate \(l^\frac8d_{\tau,k}L^4\) norm of \(e^n\) and
  \(S_K(-2\tau)e^n\). Indeed by Discrete Strichartz estimate Corollary~\ref{corDSE},
  we have
  \begin{multline}\label{enl8dL4}
    \|e^n\|_{l^\frac8d_{\tau,k+1}L^4} + \|S_K(-2\tau)e^n\|_{l^\frac8d_{\tau,k+1}L^4} 
    \lesssim \|\mathcal F_1^n\|_{l^\frac8d_{\tau,k}L^4} + 
    \|\mathcal F_2^n\|_{l^\frac8d_{\tau,k}L^4} + (K\tau^\frac12)^\frac d4 \left(
    \varepsilon^2\|G_n^1\|_{l^1_{\tau,k}L^2} \right. \\ \left.
    + \varepsilon^2\|G_n^2\|_{l^1_{\tau,k}L^2} + (K\tau^\frac12)^\frac d4 
    \varepsilon^2\|G_n^3\|_{l^{(\frac8d)'}_{\tau,k}L^\frac43}\right) .
  \end{multline}
  Consequently, by combining~\eqref{F1F2linftyL2},~\eqref{F1F2lpLq},~\eqref{Xtau}
  ,~\eqref{enlinftyL2} and~\eqref{enl8dL4}, we get 
  \begin{equation}\label{enXtaupart}
    \|e^n\|_{X_{\tau,k+1}} \lesssim 1 + \frac1\delta \left( 
    \varepsilon^2\|G_n^1\|_{l^1_{\tau,k}L^2} + \varepsilon^2\|G_n^2\|_{l^1_{\tau,k}L^2} 
    + (K\tau^\frac12)^\frac d4 \varepsilon^2\|G_n^3\|_{l^{(\frac8d)'}_{\tau,k}L^\frac43}\right).
  \end{equation}
  Finally, we can plug~\eqref{Xtau},~\eqref{G1},~\eqref{G2},~\eqref{G3d1} into~\eqref{enXtaupart}
  to get for \(d=1\),
  \begin{equation*}
    \begin{aligned}
      \|e^n\|_{X_{\tau,k+1}} &\lesssim 1 + \|e^n\|_{X_{\tau,k}} 
      (\varepsilon^\alpha k\tau)^\frac2\alpha \langle\varepsilon^\alpha k\tau\rangle^{1-\frac2\alpha} %G1
      + \delta(K\tau^\frac12)^\frac12 (\varepsilon^\alpha k\tau)^{\frac14+\frac1\alpha}
      \langle\varepsilon^\alpha k\tau\rangle^{\frac12-\frac1\alpha} \|e^n\|_{X_{\tau,k}}^2 \\
      & \qquad + \delta^2(K\tau^\frac12)^\frac34 (\varepsilon^\alpha k\tau)^\frac12 
      \|e^n\|_{X_{\tau,k}}^3.
    \end{aligned}
  \end{equation*}
  then by~\eqref{bg} and~\eqref{delta}, since \(\delta(K\tau^\frac12)^\frac12 = 
  \tau^\frac14 K^{-\frac12} = \varepsilon^\frac{4-\alpha}6\tau^\frac23\) and 
  \(\delta^2(K\tau^\frac12)^\frac34 = \tau^\frac38K^{-\frac54} = 
  \varepsilon^\frac{5(4-\alpha)}{12}\tau^\frac{17}{12}\),
  \begin{equation}\label{eXtaud1}
    \begin{aligned}
      \|e^n\|_{X_{\tau,k+1}} &\lesssim 1 + \|e^n\|_{X_{\tau,k}} 
      (\varepsilon^\alpha k\tau)^\frac2\alpha \langle\varepsilon^\alpha k\tau\rangle^{1-\frac2\alpha} %G1
      + \varepsilon^\frac{4-\alpha}6\tau^\frac23 (\varepsilon^\alpha k\tau)^{\frac14+\frac1\alpha}
      \langle\varepsilon^\alpha k\tau\rangle^{\frac12-\frac1\alpha} \|e^n\|_{X_{\tau,k}}^2 \\
      & \qquad + \varepsilon^\frac{5(4-\alpha)}{12}\tau^\frac{17}{12} 
      (\varepsilon^\alpha k\tau)^\frac12 \|e^n\|_{X_{\tau,k}}^3.
    \end{aligned}
  \end{equation}
  Similarly, in dimension \(2\), by plugging~\eqref{Xtau},~\eqref{G1},~\eqref{G2},~\eqref{G3d2} 
  into~\eqref{enXtaupart} ,
  \begin{equation*}
    \begin{aligned}
      \|e^n\|_{X_{\tau,k+1}} &\lesssim 1 + \|e^n\|_{X_{\tau,k}} 
      (\varepsilon^\alpha k\tau)^\frac2\alpha \langle\varepsilon^\alpha k\tau\rangle^{1-\frac2\alpha} %G1
      + \delta(K\tau^\frac12) \varepsilon (\varepsilon^\alpha k\tau)^\frac1\alpha
      \langle\varepsilon^\alpha k\tau\rangle^{\frac12-\frac1\alpha} \|e^n\|_{X_{\tau,k}}^2 \\
      & \qquad + \delta^2(K\tau^\frac12)^\frac32 \varepsilon^2 
      \|e^n\|_{X_{\tau,k}}^3,
    \end{aligned}
  \end{equation*}
  and by~\eqref{bg} and~\eqref{delta}, since 
  \(\delta(K\tau^\frac12) \varepsilon = \varepsilon\tau^\frac12\) and
  \(\delta^2(K\tau^\frac12)^\frac32 \varepsilon^2 =
  \varepsilon^2\tau^\frac34K^{-\frac12} = 
  \varepsilon^\frac{9+\eta}4\tau^\frac98\),
  \begin{equation}\label{eXtaud2}
    \begin{aligned}
      \|e^n\|_{X_{\tau,k+1}} &\lesssim 1 + \|e^n\|_{X_{\tau,k}} 
      (\varepsilon^\alpha k\tau)^\frac2\alpha \langle\varepsilon^\alpha k\tau\rangle^{1-\frac2\alpha} %G1
      + \varepsilon\tau^\frac12 (\varepsilon^\alpha k\tau)^\frac1\alpha
      \langle\varepsilon^\alpha k\tau\rangle^{\frac12-\frac1\alpha} \|e^n\|_{X_{\tau,k}}^2 \\
      & \qquad + \varepsilon^\frac{9+\eta}4\tau^\frac98 \|e^n\|_{X_{\tau,k}}^3 .
    \end{aligned}
  \end{equation}
  And in dimension \(3\), by plugging~\eqref{Xtau},~\eqref{G1},~\eqref{G2},~\eqref{G3d3} 
  into~\eqref{enXtaupart}, 
  \begin{equation*}
    \begin{aligned}
      \|e^n\|_{X_{\tau,k+1}} &\lesssim 1 + \|e^n\|_{X_{\tau,k}} 
      (\varepsilon^\alpha k\tau)^\frac2\alpha \langle\varepsilon^\alpha k\tau\rangle^{1-\frac2\alpha} %G1
      + \delta(K\tau^\frac12)^\frac32 \varepsilon\tau^{-\frac14} (\varepsilon^\alpha k\tau)^\frac1\alpha
      \langle\varepsilon^\alpha k\tau\rangle^{\frac12-\frac1\alpha} \|e^n\|_{X_{\tau,k}}^2 \\
      & \qquad + \delta^2(K\tau^\frac12)^\frac94 \varepsilon^2 \tau^{-\frac12}
      \|e^n\|_{X_{\tau,k}}^3 , 
    \end{aligned}
  \end{equation*}
  and by~\eqref{bg} and~\eqref{delta}, since
  \(\delta(K\tau^\frac12)^\frac32 \varepsilon\tau^{-\frac14} = 
  \varepsilon^\frac23\tau^\frac16(|\log\tau|+|\log\varepsilon|)^\frac23\)
  \(\delta^2(K\tau^\frac12)^\frac94 \varepsilon^2 \tau^{-\frac12} = 
  \varepsilon^\frac{11}6\tau^\frac{11}{24}(|\log\tau|+|\log\varepsilon|)^\frac43\)
  \begin{equation}\label{eXtaud3}
    \begin{aligned}
      \|e^n\|_{X_{\tau,k+1}} &\lesssim 1 + \|e^n\|_{X_{\tau,k}} 
      (\varepsilon^\alpha k\tau)^\frac2\alpha \langle\varepsilon^\alpha k\tau\rangle^{1-\frac2\alpha} %G1
      + \varepsilon^\frac23\tau^\frac16(|\log\tau|+|\log\varepsilon|)^\frac23 
      (\varepsilon^\alpha k\tau)^\frac1\alpha
      \langle\varepsilon^\alpha k\tau\rangle^{\frac12-\frac1\alpha} \|e^n\|_{X_{\tau,k}}^2 \\
      & \qquad + \varepsilon^\frac{11}6\tau^\frac{11}{24}(|\log\tau|+|\log\varepsilon|)^\frac43 
      \|e^n\|_{X_{\tau,k}}^3 .
    \end{aligned}
  \end{equation}
  To summarize the above discussion, define \(N_1\) such that 
  \(\varepsilon^\alpha N_1\tau = T_1 \leq T\). We can rewrite~\eqref{eXtaud1},~\eqref{eXtaud2} 
  and~\eqref{eXtaud3} as, for every \( 0 \leq k \leq N_1 \),
  \begin{equation}\label{enXtau0k}
    \|e^n\|_{X_{\tau,k+1}} \leq C_T + C_T (T_1)^{\theta} \|e^n\|_{X_{\tau,k}} 
    + C_T \tau^{\lambda_1} \|e^n\|_{X_{\tau,k}}^2 
    + C_T \tau^{\lambda_2} \|e^n\|_{X_{\tau,k}}^3
  \end{equation}
  for some \(\theta,\lambda_1,\lambda_2 > 0 \).
  Consequently, by taking \(T_1\) sufficiently small so that \(C_T T_1^{\theta} 
  \leq \frac12 \), we get that by induction for every \(k\) so that 
  \(0 \leq k \leq N_1 + 1\),
  \[ \|e^n\|_{X_{\tau,k+1}} \leq \widetilde C_T \]
  for \(\tau\) sufficiently small.

  Now let us denote \(m\) be some positive integer, and we consider \(m\) such
  that \( m \leq n \leq m+N_1\), where \(N_1\) as above.
  Let us denote 
  \( \|f^k\|_{l^p_{\tau,m,n}X} = \|(f^k)_{m\leq k\leq n}\|_{l^p_{\tau,m,n}X} \)
  for simplicity, and 
  \[
      \|e^k\|_{X_{\tau,m,n}} := \frac1\delta \|e^k\|_{l^\infty_{\tau,m,n}L^2} 
      + \frac1{\delta(K\tau^\frac12)^\frac d4} \left(\|e^k\|_{l^\frac8d_{\tau,m,n}L^4}
      + \|S_K(-2\tau)e^k\|_{l^\frac8d_{\tau,m,n}L^4}\right).
  \]
  By the same way above, we can use formula~\eqref{emn},and Remark~\ref{remF1}
  and~\ref{remF2}. 
  \begin{equation}\label{enXtaumn}
    \|e^k\|_{X_{\tau,m,n}} \leq \frac1\delta \|e^m\|_{L^\infty L^2} + C_T + C_T (T_1)^{\theta}
    \|e^k\|_{X_{\tau,m,n}} + C_T \tau^{\lambda_1} \|e^k\|_{X_{\tau,m,n}}^2 
    + C_T \tau^{\lambda_2} \|e^k\|_{X_{\tau,m,n}}^3,
  \end{equation}
  where the constant \(C_T\) are the same as~\eqref{enXtau0k}.
  By inductive hypothesis, we have \( \frac1\delta \|e^m\|_{L^\infty L^2} \leq C_T'\).
  Hence we have again 
  \[ \|e^k\|_{X_{\tau,m,n}} \leq \widetilde C_T', \]
  for some constant \(\widetilde C_T'\) and sufficiently small \(\tau\).
  We can then iterate the estimates on \([0,T_1],[T_1,2T_1],\cdots\)to finally
  obtain the result after a finite number of steps.
  %\begin{equation}
  %  \begin{aligned}
  %    \|e^n\|_{X_{\tau,k+1}} &\lesssim 1 + \frac1{\varepsilon^\beta\tau^{-\frac\gamma2}}
  %    \left( \varepsilon^2\|G_n^1\|_{l^1_{\tau,k}L^2} 
  %    + \varepsilon^2\|G_n^2\|_{l^1_{\tau,k}L^2} + (\varepsilon^{-\beta}\tau^{-\frac\gamma2})^\frac d4 
  %    \varepsilon^2\|G_n^3\|_{l^{(\frac8d)'}_{\tau,k}L^\frac43}\right),\quad
  %    d=1,2, \\
  %    \|e^n\|_{X_{\tau,k+1}} &\lesssim 1 + 
  %    \frac1{\varepsilon^\beta\tau^{-\frac\gamma2}(|\log\tau|+|\log\varepsilon|)^\frac23}
  %    \left(\varepsilon^2\|G_n^1\|_{l^1_{\tau,k}L^2} 
  %    + \varepsilon^2\|G_n^2\|_{l^1_{\tau,k}L^2} + (\varepsilon^{-\beta}\tau^{-\frac\gamma2})^\frac d4 
  %    \varepsilon^2\|G_n^3\|_{l^{(\frac8d)'}_{\tau,k}L^\frac43}\right),\quad d=3
  %  \end{aligned}
  %\end{equation}
  %\varepsilon^\beta\tau^{-\frac\gamma2}(|\log\tau|+|\log\varepsilon|)^\frac23 
  %
  \section{proof of the generalization of discrete Strichartz estimates}
  \begin{proof}[Proof of Theorem~\ref{thmDSE}]
    Since the filtered Schr\"odinger flow satisfying the following truncated
    decay estimates
    \[ \|S_K(t)f\|_{L^p} \lesssim
    \frac{K^{d(1-\frac2p)}}{1+|t|^{\frac{d}2(1-\frac2p)}}\|f\|_{L^{p'}},\]
    we can just imitate the proof the continuous case but just replace the 
    Hardy-Littlewood-Sobelev inequality by Young inequality.
  \end{proof}

  \begin{proof}[Proof of Corollary~\ref{corDSE}]
    Use Littlewood-Paley theory.
    By Littlewood-Paley inequality, then Minkowski inequality, then Discrete
    Strichartz estimate, we have
    \begin{equation}
      \begin{aligned}
        \|S_K(n\tau)u\|_{l^p_\tau L^q} & \sim \|\|(S_K(n\tau) u_k)_{k\in \mathbb{Z}}\|_{l^2}\|_{l^p_\tau L^q} \\
        &\lesssim \|(\|S_K(n\tau) u_k\|_{l^p_\tau L^q})_{k\in \mathbb{Z}}\|_{l^2} \\
        &\lesssim \|((2^k\tau^\frac12)^\frac2p \|u_k\|_{L^2})_{k\in \mathbb{Z}, 2^k \lesssim K}\|_{l^2} \\
        &\lesssim \varepsilon^{-\beta\frac2p}\|(2^{k\frac2p(1-\frac1\gamma)} \|u_k\|_{L^2})_{k\in \mathbb{Z}}\|_{l^2} \\
        &= \varepsilon^{-\beta\frac2p}\|u\|_{B^{\frac2p(1-\frac1\gamma)}_{2,2}} \\
        &\sim \varepsilon^{-\beta\frac2p}\|u\|_{H^{\frac2p(1-\frac1\gamma)}}
      \end{aligned}
    \end{equation}

    The other statement are proved by the same spirit.
  \end{proof}


  \section{some technical estimates}

  \begin{lemma}\label{lemfilter1}
    We have the following:
    \begin{itemize}
      \item For every \(p \in [1,\infty]\), there exists \(C>0\) such that for every
        \(\tau \in (0,1]\),
        \begin{equation}\label{101}
          \|\varphi_1(-2i\tau\Delta)\Pi_{\tau^{-\frac12}} f\|_{L^p} \leq
          C\|f\|_{L^p} \quad\text{for all } f \in L^p.
        \end{equation}
      \item For every \(p \in (1,\infty)\), there exists \(C>0\) such that for every
        \(\tau \in (0,1]\),
        \begin{equation}\label{102}
          \|\frac{1-\Pi_{\tau^{-\frac12}}}{2i\tau\Delta} f\|_{L^p} \leq
          C\|f\|_{L^p} \quad\text{for all } f \in L^p.
        \end{equation}
      \item For every \(s \in [0,2]\), there exists \(C>0\) such that for every
        \(\tau \in (0,1]\),
        \begin{equation}\label{103}
          \|\varphi_1(-2i\tau\Delta) f\|_{L^2} \leq
          \frac C{\tau^\frac s2}\|f\|_{L^2} \quad\text{for all } f \in L^2.
        \end{equation}
    \end{itemize}
  \end{lemma}

  \begin{lemma}\label{sobbord}
    There exists \(C>0\) such that for every \(u\in W^{1,3}(\mathbb R^3)\) with
    \(\supp \hat{u} \subset B(0,4K)\), \(K \geq 1\), we have
    \[ \|u\|_{L^\infty} \leq C (\log K)^\frac23 \|u\|_{W^{1,3}}. \]
  \end{lemma}

  \begin{thebibliography}{99}
    \bibitem{ORS21}
      {\rm A. Ostermann, F. Rousset, K. Schratz,}
      {\em Error estimates of a fourier integrator for the cubic schr\"odinger equation at low regularity.}
      Found. Comput. Math. \textbf{21}, 725--765 (2021).

  \end{thebibliography}
\end{document}
